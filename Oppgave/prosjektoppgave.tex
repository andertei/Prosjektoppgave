\documentclass[10pt,a4paper]{article}
\usepackage[latin1]{inputenc}
\usepackage[english]{babel}
\usepackage{amsmath}
\usepackage{amsfonts}
\usepackage{amssymb}
\usepackage{makeidx}
\usepackage{graphicx}
\usepackage[left=2cm,right=2cm,top=2cm,bottom=2cm]{geometry}

\author{Anders Dall'Osso Teigset}
\title{Puffer mechanism}

\begin{document}
\maketitle
%Summary
\section{Summary}
\tableofcontents
\section{Introduction}
Upon till now SF$_6$ and vacuum based technology have been dominating the medium-voltage compact switchgear market. Air insulated switchgear does exists but they are space consuming and are not applicable for use in a compact substation design. A compact substation is one of the most common designs for substations in the medium-voltage level of the distribution system. Figure \ref{fig:compact substation} displays a compact substation which can be used in the medium-voltage distribution system. As figure \ref{fig:compact substation} displays there is a limited amount of space available for the switchgear. Therefore the main challenge for a air insulated switchgear for this application is to meet the space requirements set by the amount of space available in the compact substation. 
\end{document}