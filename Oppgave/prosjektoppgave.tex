\documentclass[10pt,a4paper]{article} %hvor stor skrift?
\usepackage[latin1]{inputenc}
\usepackage[english]{babel}
\usepackage{amsmath}
\usepackage{amsfonts}
\usepackage{amssymb}
\usepackage{makeidx}
\usepackage{graphicx}
\usepackage{hyperref}
\usepackage[left=2cm,right=2cm,top=2cm,bottom=2cm]{geometry}
\usepackage{float}

%symbolliste


\usepackage{makeidx}
\makeindex
%symbolliste slutt

\author{Anders Dall'Osso Teigset}
\title{PUFFER MECHANISM SKRIV I STORE BOKSTAVER}

\begin{document}
\maketitle
%SummaryNyttige pdf-filer/SF6conduct.pdf
\newpage
\section{Summary}
\newpage
\tableofcontents
\newpage
\section{Introduction}
A load break switch should be able to interrupt currents less or equal the maximum load current in a distribution system. When interrupting a current the two contacts which a switch consists of start separating producing a gap between each other. Normally the current is not interrupted by this and an electrical arc ignites and burns in the contact gap \cite{bib:HVEbreak}. The arc consists of plasma, which is a mixture of negative and positive ions as well as electrons. Due to the energy dissipation produced by the arc the temperature in the plasma is very high. The electrical conductivity of the plasma channel is dependent of the temperature produced by the arc. When a high current is flowing it is almost a perfect conductor, but at low temperatures the conductivity is much poorer.

An AC-current have a natural zero crossing, as the current approaches this point the arc will start cooling and extinguish if properly cooled at this point of the power cycle. The working principle of a switchgear is to cool the arc sufficiently when the current approaches zero and then quench the arc when the current is zero. At this moment the current is interrupted and a voltage builds up across the open contacts. This voltage is called the recovery voltage. The steepness and amplitude of the recovery voltage \cite{bib:HVEbreak} and the design of the switchgear will decide if a new arc ignites between the contacts after the current zero. If an arc re-ignites ether by a thermal or a dielectric breakdown the interruption has failed. 

Upon till now SF$_6$ and vacuum based technology have been dominating the compact medium-voltage switchgear market. Air insulated switchgear does exists but they are space consuming and are not applicable for use in a compact substation design. A compact substation is one of the most common designs for substations in the medium-voltage level of the distribution system. Figure \ref{fig:compact substation} displays a compact substation which can be used in the medium-voltage distribution system. The switchgear is a module that can be detached from the compact substation and removed trough its front panel. As figure \ref{fig:compact substation} indicates there are a limited amount of space available for the switchgear. Therefore the main challenge for an air insulated switchgear design for this kind of application is set by the amount of space available in the compact substation.

\begin{figure} [h]
\centering
\includegraphics[scale=0.5]{Bilder/Introduction/general_substation.jpg}
\caption{Compact substation with open front panel \cite{bib:comSub}} \label{fig:compact substation}
\end{figure}

The main choice of interrupting media in switchgear has been SF$_6$ since its discovery in the 1970ies. The gas exhibits many properties which is well suited for an isolation gas. It is highly electron negative which gives it good dielectric strength and arc-interrupting capabilities. The breakdown voltage of SF$_6$ is almost three times higher than air at atmospheric pressure \cite{bib:SF6PI}. It has excellent heat transfer properties as an interrupting medium \cite{bib:SF6PI}. The gas also reforms itself when dissociated under the high temperature conditions in an electrical arc. SF$_6$ produces no polymerization, carbon or other conductive deposits during arcing. It is also chemically compatible with most insulation materials and conductive materials \cite{bib:SF6PI}. The gas is also well suited for use in low temperature environments since its boiling point is fairly low even at high pressures. The gas in its stable form is nontoxic, non-flammable and non-explosive. It is also thermally stable and do not decompose at normal operating temperatures for a closed switch \cite{bib:SF6PI}. SF$_6$ based switchgear tends to be cheap to produce relative to other designs, and the gas itself is also affordable at today prices.


SF$_6$ have some disadvantages, when exposed to electrical discharge or arcing it forms highly toxic and corrosive compounds \cite{bib:SF6PI}. It is also hard to remove non-polar contaminants like air and its breakdown voltage is sensitive to water vapour and conductive particles. However it biggest downside is probably that it is an effective infrared absorber. This makes it a strong greenhouse gas \cite{bib:SF6PI} and it is regarded to be almost $20000$ times as potent as CO$_2$.

0.4 \% of the greenhouse gas emissions in Norway is due to SF$_6$ and is mainly used in switchgear or other high voltage equipment \cite{bib:KlimaKur2020}. In Norway the use of SF$_6$ is regulated through a voluntary contract between the environmental department and the end user, mainly the power companies \cite{bib:KlimaKur2020}. If a compatible air based switchgear design is released on the market it can be assumed that the power companies will be interested in the new technology. It is also possible that the government will restrict the use of SF$_6$ gas if the private sector do not follow the guidelines of the environmental department.

A test switch have been developed which allows adjustment of many design parameter such as nozzle and contact geometry, contact movement and gas pressure. The contacts consist of one female contact, or tulip and one male contact, or pin. The tulip is immovable, while the pin can be opened by a spring trigged by an electromagnet. Previous results have illustrate that most of the successful interruptions have occurred when the pin is outside the nozzle \cite{bib:CIAMVLBS}. Cold air simulations done by Nina Sasaki Aanensen of the different geometries have stated that volumetric flow of air is much lower when the pin is inside the nozzle. The results from previous testing and cold air simulations might suggest that the pin is clogging the nozzle and preventing a good air flow. Therefore a new nozzle geometry has been designed with a doughnut area, the area between the nozzle and the pin, which is bigger than the area of the pin. This is assumed to give a volumetric flow of air that is almost the same when the pin is inside the nozzle as well as outside. The theory has been verified by cold air simulations. However the clogging effect generated by an arc has not been taken into account and it is expected that this will have an impact on the volumetric flow. Two nozzles have been constructed of the new nozzle geometry one with an doughnut area of $44 \ mm^2$ and the other one with an area of $66 \ mm^2$. It is expected that the speed of the air will be constant and the volumetric flow will increase with the doughnut area. This will give an opportunity to test which of these parameters that influence the current interruption capabilities the most.

Several different nozzle geometries are going to be tested and it is assumed that the chance for a successful current interruption inside the nozzle will be approximately the same as outside the nozzle. Since the size of the arc compared to the pin will vary between the different nozzle geometries and current amplitudes, a clogging effect is assumed to occur and might be revealed as an lower interruption success rate for certain geometries.

\newpage
\section{Theory}
\subsection*{Symbol list}
\begin{description}
\item{$U_{recovery}$} \ -- \ Voltage that appears across the terminals after current interruption.
\item{$U_{left}$} \ -- \ Voltage on the left hand side of the switch after current interruption.
\item{$U_{right}$} \ -- \ Voltage on the right hand side of the switch after current interruption.
\end{description}
\newpage
\subsection{Switchgear design}
\subsubsection{General design} \label{sec:genDes}
Switchgear can be divided into four main categories:
\begin{itemize}
\item Disconnector Switch
\item Load Break Switch
\item Circuit Breaker
\item Earthing Switch
\end{itemize}
This report will focus on the load break switch (LBS) design. A LBS is a switchgear design to be able to interrupt currents with magnitude that is equal or less than the rated maximum continuous current in a transmission system. A LBS must fulfil the following demands to meet the requirements of the application area:

\begin{itemize}
\item When closed:
	\begin{itemize}
		\item It must act as an perfect conductor.
		\item Be capable to interrupt any currents that may arise, without generating too high over-voltages. 
	\end{itemize}
\item When open:
	\begin{itemize}
		\item It must be an perfect isolator.
		\item Be able to close without welding the contacts together, event under short-circuit conditions.
	\end{itemize}
\end{itemize}

In a LBS it is important to assure that the switch in closed position acts as a perfect conductor with low contact resistance. This will minimize electrical losses in the switch and therefore minimize economical losses. Copper or aluminium is ideal materials and are often used to ensure that this aspect of the contact is met. Sometimes the contact surface is plated with tin, gold, silver or platinum to ensure a low contact resistance between the contact plates. The main problem with electrical losses in the switch is heat generation which may speed up metal creeping processes and other ageing related processes in the switch. Contact plates made out of aluminium are especially vulnerable to creeping.
 
To withstand the harsh conditions that occurs when an arc burns between the contacts the contact material has to meet strict requirements. It has to tolerate high temperatures, arc erosion, welding and other stresses that may apply when closing or opening an energized contact. Aluminium and copper is not fit for these tasks since they will melt or erode from stresses of an arc. It is common to use composites of metals with good electrical conduction with heat resistant oxides. For high current and voltage switches it is usual to use a composite of silver or copper together with tungsten or tungsten carbide. These materials are very resistant against heat, erosion and welding, but they have a high resistance. Therefore it is common on breakers above $70 \ kV$ to have two sets of contacts, one arcing contact and one main contact. The main contact is the first contact to open and the last one to close. This is to ensure that an arc do not start to burn between the contact, which makes it possible to use aluminium or copper as contact materials. The arcing contact is the last contact to open, and the first one to close. This will ensure that the arc burns between the arcing contact and not between the main contact. This contact arrangement is also common in the compact LBS systems available today, and many other common switchgear designs.

To quench the arc several mechanisms and interrupting mediums can be applied. In today's compact LBS systems based on SF$_6$ the puffer mechanisms is commonly used. Details regarding arc quenching is explained in section \ref{sec:InterruptCurrent}. To obtain a successful interruption of an arc it is possible to cool the arc and interrupting gas down, as well as blow away charged particles and metal oxides between the contact points. This is the main goal with the puffer mechanism. As SF$_6$ entered the industry in the 1960s it was in the form of dual pressure breakers \cite{bib:HVEbreak}. The SF$_6$ based dual pressure breakers had one high pressure chamber, and one low pressure chamber. A valve from a high pressure chamber opens during opening operations, generating a high speed SF$_6$ blast, guided by a nozzle, to hit the arc burning between the contacts in the low pressure chamber. This design have two major disadvantages when using SF$_6$ instead of air. The switch requires heating to maintain the high pressure in the high pressure chamber so that the SF$_6$ gas do not condensates. It also uses a compressor to pump the gas from the low pressure chamber to the high pressure chamber. This adds to the complexity to the switchgear and may result in more maintenance. The double pressure design was replaced with the single pressure design. The single pressure design has only one pressure chamber with low pressure except during interruptions when the chamber it self becomes a high pressure reservoir \cite{bib:HVEbreak}.

The single pressure SF$_6$ switchgear is using the puffer or the self-blast mechanism to quench an arc, or in some cases a combination with both. Common for both puffer and self-blast mechanisms are that they do not use a compressor to generate the gas flow, but uses the energy stored in the switching mechanisms or generated from the blast itself to interrupt the arc \cite{bib:HVEbreak}. These mechanisms works the same way for both circuit breakers and load break switches, but will in a load break switch be smaller and less complex. This is because the current amplitudes are smaller and therefore a lower pressure is needed to obtain a successful interruption.

The puffer mechanism is based on a piston to generate a flow of gas in the switchgear to extinguish the arc. A typical design of this system is by using a fixed piston integrated as a part of the contact design \cite{bib:HVEbreak}. A gas reservoir trapped between the arcing contact and the piston is pushed out as the contact moves apart for each other. Figure \ref{fig:CircutBreakPuff1} displays a typical interruption sequence of a puffer design breaker.


\begin{figure} [h]
\centering
\includegraphics[scale=0.8]{Bilder/Theory/CircutBreakPuff1.png}
\caption{Interruption sequence in a puffer breaker \cite{bib:HVEbreak}} \label{fig:CircutBreakPuff1}
\end{figure}

When the breaker is closed as seen in figure \ref{fig:CircutBreakPuff1}a there is a gas volume \textit{(V)} trapped between the piston \textit{(P)} and the arcing contact \textit{(8) and (6)}. When the movable part of the arcing contact is pulled down the volume decreases because of the fixed piston. Then an increase in pressure due to compression of the gas occurs. Figure \ref{fig:CircutBreakPuff1}b illustrate that the main contact is open, and that the current now only flows through the arcing contact.

The next stage of the interruption sequence is pointed out in figure \ref{fig:CircutBreakPuff1}c. The arcing contacts have now separated and an arc \textit{(A)} has ignited between the contacts. The pressurised gas that previously was trapped between the piston and the arcing contact is now released. The gas flow is guided by a nozzle \textit{(9)} that is fixed to the movable arcing contact. The gas flow will cool down the arc and blow away charge carriers between the contact plates. If a sufficient gas flow is obtained the arc will not re-ignite after current zero and neither extinguish before current zero, so that current chopping is avoided.

The gas flow is partially dependent on the cross-section of the arc, which again is dependent on the current amplitude. A large current resulting in an large arc may block the hole in the nozzle preventing a gas flow. This is called current clogging and may occur for certain nozzle designs at high current interruptions. In such an event the pressure in the gas reservoir will increase further due to compression from mechanical moment of the arcing contact and thermal expansion in the gas because of heating from the arc. When the current amplitude approaches zero its cross-section will decrease and the clogging effect will end. This will result in a powerful gas blast onto the arc, as indicated in figure \ref{fig:CircutBreakPuff1}d. For smaller current amplitudes the arc cross-section is smaller and a blocking effect does not occur in the same extent. This generates a less intense gas flow, preventing current chopping.
 
\begin{figure} [h]
\centering
\includegraphics[scale=0.4]{Bilder/Theory/selfBlast.png}
\caption{Expulsion chamber in a breaker using the self-blast mechanism \cite{bib:CBAC}} \label{fig:selfBlast}
\end{figure}

The self-blast, or third generation breaker, was developed with the goal of reducing mechanical power of the operating system, making it cheaper and less complex. Figure \ref{fig:selfBlast} illustrates the working principle of a breaker using self-blast to interrupt an arc. The difference between self-blast and puffer mechanism is that the puffer mechanism increases the pressure by reducing the volume. Rather than the self-blast design which have a constant volume relying on a raise in temperature to increase the gas pressure \cite{bib:CBAC}. The self-blast design uses the heat generated from an arc burning between the arcing contacts to interrupt the current. The gas expands as it is heated by the burning arc, this increase in pressure leads to a gas flow on the arc, which cools it down leading to the quenching of it.

There are some disadvantages with the self-blast principle when compared to the puffer mechanism. The self-blast have a lower dielectric strength due to hot gas between the contacts after CZ. This gives a higher chance of re-ignition since hot gas has a higher conductivity than cold gas \cite{bib:HVEbreak}. It is also not well suited to break smaller currents. This is because the arc is less intense and therefore do not heat the gas sufficiently to create a strong enough blast. Because of this it is common to combine self-blast and puffer mechanism in a hybrid design, so that it can handle both small and large currents \cite{bib:HVEbreak}. A compact LBS design using air as interrupting medium will probably rely on a puffer or a hard gas design. This is because of the smaller currents an LBS is facing compared to a circuit breaker.


\subsubsection{Test switch}
A new medium voltage laboratory for load break switch development have been designed with the possibilities to vary important design parameters for a load break switch. When the industry develops switchgear technology they tend to alter several different parameters at the same time. For instance, if they wish to alter the speed of the airflow in a puffer LBS, they might increases the separation speed of the contacts, since these functions often are linked in a puffer design. This will also result in a higher pressure not only and not only a higher volumetric flow. This will make it hard to conclude which parameter that was the most critical for a successful interruption. Therefore a laboratory switch have been design where each of the different parameters can be adjusted without changing other critical parameter. This will make systematic researching possible and optimisation of certain designs criteria, like dimensions, will be a lot simpler.

\begin{figure} [h]
\centering
\includegraphics[scale=0.5]{Bilder/Theory/SchematicTestSwitch.png}
\caption{Test switch} \label{fig:testSwitch}
\end{figure}

Figure \ref{fig:testSwitch} shows the physical appearance of the switch and how the different parameter can be adjusted. BLA BLA TAKE A PICTURE OF THE SWITCH, WRITE SOME NUMBERS ON IT. TAKE ABOUT THEM AND WITCH PARAMETERS THAT CAN BE ADJUSTED.

\begin{figure} [h]
\centering
\includegraphics[scale=0.5]{Bilder/Theory/SchematicTestSwitch.png}
\caption{Tulip, nozzle and pin, L and D is the length and inner diameter of the nozzle, x is contact position. Dimensions are in millimetres.   \cite{bib:CIAMVLBS}} \label{fig:SchematicTestSwitch}
\end{figure}

Figure \ref{fig:SchematicTestSwitch} display the former nozzle geometries and it can be clearly seen from this picture that the doughnut area is smaller than the tulip area. It is possible that this will give a clogging effect when the pin is inside the nozzle and result in a poorer airflow and interrupting capabilities for this situation. In figure \ref{fig:differentGeometries} the different contact geometries that have been made is shown. Geometries a1, b1, and c1 have been tested and the results showed that most of the successful interruptions happen when the pin was outside the nozzle. It is possible that this is the result of a clogging effect that takes place between the nozzle and pin. Therefore geometries b2 and c2 was created. The geometries is similar to b1 and c1 since they have the same nozzle area, respectively $44 \ mm^2$ and $66 \ mm^2$. The volumetric airflow will however be smaller for b2 and c2 because it have been created a choking point in the tulip. 

\begin{figure} [h]
\centering
\includegraphics[scale=0.6]{Bilder/Theory/differentGeometries.png}
\caption{Different contact geometries \cite{bib:CIAMVLBS}} \label{fig:differentGeometries}
\end{figure}

It is expected that b2 and c2 will need a higher pressure than b1 and c1 to perform a successful interruption, due to the lower airflow. However it is expected that they will have an equal successrate no matter if the pin is inside the nozzle or outside the nozzle.
\newpage
\subsection{Interrupting currents} 
\subsubsection{Phenomenological description of a typical current interruption} \label{sec:InterruptCurrent}
A interrupting sequence in an AC power system follows a typical pattern. During normal service the contacts are closed and current is flowing through the switchgear without electrical losses. When a command signal to interrupt the current is sent to the breakers driving mechanism the contacts are set to motion. A gap opens between them and is filled with an interrupting medium. An arc usually forms between the contacts and the current continues to flow trough the breaker. This arc mainly consists of plasma which is composed of electrons and other charge carriers like positive and negative ions. Due to energy dissipation in the arc the temperature in the plasma is generally high, but strongly dependent on the amplitude of the electrical current passing through the breaker \cite{bib:HVEbreak}. The system load is the main factor when assessing the electrical current flow through the breaker, the effect from the arcing voltage is often disregarded. Because of the limited time span when assessing breaking operations the sinus curve of the current at a given time is needed to be taken account for. Plasma with an high temperature tends to have a higher electric conductivity than plasma with an lower temperature. Electrical conductivity in an plasma channel is also strongly depend on the medium the plasma is burning in, the difference between air and SF${_6}$ will be discussed in section \ref{sec:airandsf}.

The arcing voltage, defined as the voltage drop over the arc, is dependent on the temperature and the cross-section of the arc \cite{bib:HVEbreak}, since these are depending on the current the arcing voltage can be regarded as constant when the current exceeds a certain level. This is further explained in section \ref{sec:elarcs} . When the current approaches its zero crossing (CZ) the energy dissipation in the arc decreases and the plasma cools down. With different cooling mechanisms as mention in section \ref{sec:genDes} the breaker cools down the arc. If done sufficient the electric conductivity of the interrupting medium becomes low, and it behaves as an isolator. This will quench the arc as the current amplitude reaches zero.

The moment after the arc have been quenched a voltage builds up over the contacts, this is called the recovery voltage \cite{bib:HVEbreak}. If a successful interruption is to occur, a re-ignition of the arc must be avoided as the recovery voltage increases. The probability of an re-ignition is set by the steepness and amplitude of the recovery voltage \cite{bib:HVEbreak}. There is two different kinds of re-ignition: thermal and dielectric re-ignition. Thermal re-ignition take place right after CZ and is mainly dependent on the recovery voltage's steepness. As the recovery voltage rises and a thermal re-ignition is avoided a dielectric re-ignition may occur. This kind of re-ignition is largely dependent on the amplitude of the recovery voltage. The recovery voltage can be mathematically expressed as described in equation \eqref{eq:U_rec}. This formula presents a opportunity to analyse the steepness and amplitude of different interrupting scenarios mathematically.

\begin{equation} \label{eq:U_rec}
u_{recovery}=u_{left}-u_{right} \cite{bib:HVEbreak}
\end{equation} 

Where $u_{left}$ is the voltage on the left hand side of the breaker and $u_{right}$ is the voltage on the right hand side.

\subsubsection{Voltage drop in a static electric arc} \label{sec:elarcs}

Plasma is generated when gas or metal vapour is heated to very high temperatures, at a certain point the molecules in the gas decomposes to ions and free electrons. This mixture is called plasma, and makes up most of the components in an electrical arc \cite{bib:HVEbreak}. The following section describe the properties of an arc burning in atmospheric pressure or above. Arcs in vacuum behaves different, and will not be featured in this report.

When working with arcs it is common to differentiate between dynamic and static arcs. The static arc is established during switching operations in a DC power system after the transient phenomenas have faded out. An dynamic arc is an arc that occurs in AC power systems and during the transient part of a DC interruption, or when the cooling of the arc varies with time, like in most puffer designs. Even though all arcs in an AC system is to be regarded as dynamic arcs, it is possible to regard them as static during a short period of time \cite{bib:HVEbreak}.

Figure \ref{fig:staticArcChar} displays the static arc characteristic for an arc burning in a gas filled gap. The figure is only valid for pressure levels equal or above atmospheric pressure. The static arc characteristic gives a relation between the arcing voltage and the electrical current flowing in the gap. The values on the axis is only approximations and varies with the gas type and contact material. The length of the gap is also an important factor the magnitude of the arcing voltage. 

\begin{figure}[H]
\centering
\includegraphics[scale=1]{Bilder/Theory/staticArcChar.png}
\caption{Static arc characteristic  \cite{bib:HVEbreak}.} \label{fig:staticArcChar}
\end{figure}

At low currents the arcing voltage is high, but decreases rapidly with increasing currents. For higher currents the voltage drop is approximately constant, until the current reaches a certain level and the arcing voltage starts to increase. The difference between air and SF$_6$ as an interrupting medium will be discussed further in section \ref{sec:airandsf}, but on a general basis it is likely to assume that SF${_6}$ have a lower arcing voltage if compared to air for a certain breaking operation. This implies that the energy dissipation in air will be higher, especially when breaking smaller currents where the arcing voltage difference is greatest. This is highly relevant for an LBS since most of the switching duties occurs at smaller currents.

An static electrical arc might be regarded as divided into three regions \cite{bib:HVEbreak}.
\begin{description}
\item[Chatode region] The voltage drop, $V_c$, is usually around 20 V.
\item[Arc column]	There is a constant electric field in this region, typical 10 V/cm.
\item[Anode region] The voltage drop, $V_a$, is usually around 3 V.
\end{description}

This relationship is illustrated in figure \ref{fig:potDisArc}.
\begin{figure}[H]
\centering
\includegraphics[scale=0.8]{Bilder/Theory/potentialDistArc.png}
\caption{Cross-section of a stationary arc and the corresponding potential distribution \cite{bib:HVEbreak}.} \label{fig:potDisArc}
\end{figure}

For short arcs the voltage drop will mostly occur close to the cathode, and some what at the anode. In longer arcs more of the voltage drop will occur in the arc column itself.

\subsubsection{Electrical conductivity in an arc} \label{sec:eleCondArc}
Gasses have the ability to be perfect isolators as well as good conductors, mainly depending on the gas temperature. This is due to charged particles and electrons created by dissociation of the molecules in the gas. Air is a mixture of several gasses but might be simplified to consist mostly of nitrogen (N$_2$). In figure \ref{fig:condAir} the conductivity of air can be observed.   

\begin{figure}[H]
\centering
\includegraphics[scale=0.8]{Bilder/Theory/airConduct.png}
\caption{Electrical conductivity of air at atmospheric pressure \cite{bib:HVEbreak}.} \label{fig:condAir}
\end{figure}

The steep increase in conductivity can mainly be explained by the dissociation process and ionization of N$_2$ due to temperature increase. The particle density of nitrogen as it dissociates due to high temperature in the gas is illustrated in figure \ref{fig:Ndensi}. When figure \ref{fig:Ndensi} is compared to figure \ref{fig:condAir} a connection between temperature and the rapid decline of N$_2$, generation of the positive ion N$^+$ and the steep increase in conductivity of air is clearly presented.

\begin{figure}[H]
\centering
\includegraphics[scale=0.8]{Bilder/Theory/particleDensNit.png}
\caption{Particle density for different dissociation products of nitrogen as a function of temperature \cite{bib:HVEbreak}.} \label{fig:Ndensi}
\end{figure}

From figure \ref{fig:Ndensi} the electron positive effect of N$_2$ is also indicated via the generation of N$_{2}^{+}$ molecules. From table \ref{tab:thermalIonisation} the thermal ionisation energy for some gases are presented. This points out that N$_2$ have a significant lower ionisation energy than SF$_6$, and it gives away electrons more easily. However sulphate and fluorine have a much lower ionisation energy, which is products of the dissociation of SF$_6$. This indicates that when SF$_6$ first is dissociated the ionisation and conductivity of the gas rapidly increases. This is also pointed out in figure \ref{fig:SF6densi} with indicates the particle density of SF$_6$.

\begin{table}[H]
\center
\caption{Thermal ionisation energy for some gases \cite{bib:HVEbreak}.}
\begin{tabular}{|l|c|c|}
\hline 
Particle type & Single ionisation [eV] & Double ionisation [eV] \\ 
\hline 
Air & 16.3 &  \\ 
\hline 
N$_2$ & 15.8 &  \\ 
\hline 
N & 14.5 & 44.1 \\ 
\hline 
O$_2$ & 12.5 &  \\ 
\hline 
SF$_6$ & 19.3 &  \\ 
\hline 
S & 10.4 & 33.8 \\ 
\hline 
F & 17.4 &  \\ 
\hline 
\end{tabular} 
\label{tab:thermalIonisation}
\end{table}

\begin{figure}[H]
\centering
\includegraphics[scale=0.5]{Bilder/Theory/particleDensSF6.png}
\caption{Particle density for different dissociation products of SF$_6$ as a function of temperature \cite{bib:IPSF6AQM}.} \label{fig:SF6densi}
\end{figure}

\begin{figure}[H]
\centering
\includegraphics[scale=0.5]{Bilder/Theory/SF6Conduct.png}
\caption{Electrical conductivity of SF$_6$ at atmospheric pressure \cite{bib:IPSF6AQM}.} \label{fig:condSF6}
\end{figure}

%Noe av dette egner seg kanskje bedre som disusjon
In figure \ref{fig:condSF6} the electrical conductivity of SF$_6$ as a function of temperature is presented. If the conductivity of air and SF$_6$ is compared it is possible to observe several similarities in the conductivity when compared to different temperatures. At high temperatures, when both gases are fully ionized, the conductivity is high, almost in the same range as metals. The transaction between the isolating and the conducting stage is quick. If the particle density of the two gases is taken in to account it is possible to observe that the decomposition of SF$_6$ occurs at a lower temperature than nitrogen. This can indicate that the transaction form isolation to conducting take place a bit faster for SF$_6$ than air. In a LBS this might slightly influence the interruption capabilities, but it is not the electrical conductivity that is the main difference and challenge between SF$_6$ or air as an interrupting medium. Both SF$_6$ and air have fairly good electrical conductivity profiles for interrupting currents.
%Noe av dette egner seg kanskje bedre som disusjon
However when the current amplitude approaches zero and the gases recombines, SF$_6$ have one major advantage that air do not possess. Decomposed SF$_6$ consists of a high concentration of ionized fluorine, both F$^{+}$ and F$^{++}$. These particles are considered to be highly oxidative, which means that they will attract electrons. In the moment right after CZ there are a lot of free electrons in the gap between the contact plates. It is essential to remove these quick to avoid thermal re-ignition of the electrical arc. In an SF$_6$ based switchgear many of these free electrons are absorbed by the ionized fluorine. In air oxygen have this effect, but the concentration of ionized oxygen is far lower than fluorine. This is one of the reasons that thermal re-ignition is a larger problem for an air based compact LBS and not so huge in a SF$_6$ based system.
   
\subsubsection{Heat transport in an arc} \label{sec:HeatTransport}
There are several different conductive mechanisms in an electrical arc. The effects of these mechanisms vary with temperature, and therefore the heat transport in the arc is strongly dependent upon the temperature. In figure \ref{fig:tempConGas} several common interrupting gases thermal heat conductivity is compared to each other as a function of temperature.

\begin{figure}[H]
\centering
\includegraphics[scale=0.8]{Bilder/Theory/thermalCond.png}
\caption{Thermal conductivity as a function of temperature \cite{bib:HVEbreak}.} \label{fig:tempConGas}
\end{figure}

As illustrated in figure \ref{fig:tempConGas} the thermal conductivity of air differ quite much from the one of SF$_6$. Due to the nature of different stages in current interruption it is desired to use a gas that have a thermal conductivity that suites the different stages right. 

When the current amplitude is rising or is high it is preferred that the thermal conductivity is low. This means that the plasma channel do not heat its surrounding but mainly keep the dissipated energy stored in it self. This will result in a temperature rise in the plasma channel, and a relatively small increase in the surroundings. As explained in section \ref{sec:eleCondArc} a high arc temperature will result in high conductivity in the arc, which gives a low arcing voltage. If the thermal conductivity is high in this region, heating of the surrounding system will occur this should be avoided mainly to reduced the amount of dissociation of extra interrupting medium. This might result in a slower transaction between the electrical conductive and isolation stage, resulting in a higher chance of re-ignition.

At the moment right before CZ it is an advantage that the thermal conductivity of the gas is high. This will result in a fast cool-down time of the plasma channel since both the current amplitude is decreasing and the energy stored in the arc now is permitted out in the surroundings. A gas with high thermal conductivity in this stage of the interruption process will be able to recombine from a ionized and highly conductive state to a isolator fast, making it harder for a thermal re-ignition to occur. This is because of the quick cooling of the plasma channel. In gases where the thermal conductivity is low the cooling mechanisms is of great importance since a quick recombination of ionized gas do not occur in the same manner as when the medium is quickly cooled. Therefore removal of hot gas and charge carriers must be done differently, this is described in detail in section \ref{sec:genDes}.

\begin{figure}[H]
\centering
\includegraphics[scale=0.3]{Bilder/Theory/tempZonesArc.png}
\caption{Radial temperature distribution in a plasma channel \cite{bib:TDCIGBB}.} \label{fig:tempDist1}
\end{figure}

The temperature distribution in an plasma channel can be divided into three sections \cite{bib:TDCIGBB} as illustrated with figure \ref{fig:tempDist1}. Zone 1 is the highly conductive arc core and also the zone with the highest temperature. Zone 2 acts as an energy buffer during the decay of the arc while zone 3 is the cold gas surrounding the arc. When using cooling-mechanisms to quench the arc it is primary the second zone of the temperature profile that is cooled. The first zone's temperature will mainly be dependent on the current passing through the arc and will not be influenced by the cooling mechanism is the same degree. If the cooling is sufficient the energy stored in zone 2 when the arc approaches CZ is low and there its effect as a energy buffer is reduced, resulting in a rapid decline in temperature in the arc core as the current approaches zero. Therefore the interrupting mediums ability to transport energy is important when investigating efficient cooling methods. As figure \ref{fig:tempConGas} have pointed out SF$_6$ have the ability to transfer heat between zone 1 and 2 fast in the right temperature range compared to the interrupting sequence, air have a poorer ability to do this.

\begin{figure}[H]
\centering
\includegraphics[scale=0.8]{Bilder/Theory/plasmaChannel1.png}
\caption{Radial temperature distribution in a plasma channel \cite{bib:HVEbreak}.} \label{fig:tempDist2}
\end{figure}

In figure \ref{fig:tempDist2} it is illustrated how the temperature distribution varies with the electrical current. Due to radiation losses in the arc the temperature have a upper limit about 20 000 K to 30 000 K, at this point the cross-section of the arc will increase rather than the temperature \cite{bib:HVEbreak}. However it is not common for a LBS to experience these temperature ranges and its temperature distribution will mainly be in the lower current part of the figure.

%Dette hører kanskje til i diskusjon?


\subsubsection{The difference between air and SF$_6$ as interrupting medium} \label{sec:airandsf}
hva sskjer med ledeevnen naar den nearmer seg null, er hoey eller stigende? Dette maa diskuteres.
There is a curve located 4-13 ionisation occurs at the sudden changes. Compare to a similar for SF6.

Statisk lysbue karakteristikk: Luft vill ha noe hoyere lysbue spenning, spesielt for lavere strømnivaaer. Men dette er en veldig generel sluttning. Det vil avhengige av mange faktorer, som kjoling, dissosiering, elektrodemetaller. Dessuten er dette en statisk betraktning og ikke direkte overforbar til dynamiske sitvasjoner som ac brytning er. Det er ikke nødvendigvis slik at en høyere lysbue spenning resulterer i mer energi avgitt til omgivelsene i form av varme. Det kan like gjerne resultere i at bare plasma kanalen blir varm, dette er avhengig av de termiske egenskapene til luft og SF6 som re en viktig sammenligning og det er her SF6 har en virkelig fortrinn.

DETTE UNDERKAP SKAL NOK INN I DISKUSJON

\newpage
\subsection{Environmental impacts of SF$_6$ from electrical power industries}
Locally the environmental impact from SF$_6$ based switchgear is low. If contaminants, like air and water vapour, is present in the switchgear highly toxic and corrosive compounds might form when the SF$_6$ is subjected to electrical discharges \cite{bib:SF6PI}. Usually this will only be harmful for the equipment, but can result in danger for personnel in close proximity of the switchgear or inside a building containing a large SF$_6$ insulated system during a blow-out. However SF$_6$ is considered to be a safe and stable compound for use in switchgear. The main environmental impact of SF$_6$ is based on its potential as an green house gas.

SF$_6$ is a highly efficient infrared absorber, this combined with its chemical inertness makes it one of the strongest greenhouse gasses \cite{bib:SF6PI}. In Norway SF$_6$ makes up 0.4 \% of the total greenhouse gas emissions when measured in CO$_2$ equivalents \cite{bib:KlimaKur2020}. Due to the greenhouse gas potential of SF$_6$ this is a fairly small amount of gas, and the emissions is mostly from leakage in high voltage equipment. In Norway the use of SF$_6$ is regulated through a voluntary agreement between the user group
and the environmental department \cite{bib:KlimaKur2020}. The user group consists of almost all major hydropower companies and major electricity distribution companies.


\begin{figure}[H]
\centering
\includegraphics[scale=0.4]{Bilder/Theory/consentrationSF6.png}
\caption{Average SF$_6$ concentration in the atmosphere \cite{bib:consSF6}.} \label{fig:conSF6}
\end{figure}

Because of the increase in commercial of use SF$_6$ since the 1970s the production of the gas have steadily increased. This have resulted in a rise of the SF$_6$ concentration in the atmosphere from barely measurable quantities in the 1980s \cite{bib:SF6PI} to relatively high quantities now. In figure \ref{fig:conSF6} the concentration of SF$_6$ in the atmosphere from 1998 and upon till today is indicated. Awareness of SF$_6$ as a potent greenhouse gas have increased in the resent years and as figure \ref{fig:SF6EmissNor} illustrates the emission have been reduced by almost a half in the period from 2000 to 2003. The voluntary agreement was signed in 2002 \cite{bib:regSF6Miljo} and resulted in a methodological change, as the new reporting practice was introduced in 2003. The industry also reports a significant improvement in the handling of SF$_6$ \cite{bib:StatSF6}. It is these two major changes that probably is the reason for the huge drop in SF$_6$ emissions in Norway. 

\begin{figure}[H]
\centering
\includegraphics[scale=0.6]{Bilder/Theory/emissionsSF6Norway.png}
\caption{Actual emission of SF$_6$ in Norway \cite{bib:StatSF6}.} \label{fig:SF6EmissNor}
\end{figure}

In 2005 202 tonnes of SF$_6$ was installed in high voltage switchgear and circuit breakers within the user group members and in addition an estimated 2 tonnes was installed non-members equipment \cite{bib:StatSF6}. Medium voltage switchgear is mainly used by distribution companies and the installed capacity is estimated to be 60 tonnes in 2000 \cite{bib:StatSF6}. The user group only control half of the installed capacity, the rest is controlled by non-members. 

\begin{table}[H]
\center
\caption{Leakage rates from product containing SF$_6$ \cite{bib:StatSF6}.}
\begin{tabular}{|c|c|c|}
\hline 
\textbf{Product emission source}
 & \textbf{Yearly rate of
leakage (per cent)}
 & \textbf{Product lifetime
(years)}
 \\ 
\hline 
Gas-insulated switchgear (GIS)
 & 1 & 30 \\ 
\hline 
Sealed medium voltage switchgear
 & 0.1 & 30 \\ 
\hline 
Electrical transformers for
measurements
 & 1 & 30 \\ 
\hline 
Sound-insulating windows
 & 1 & 30 \\ 
\hline 
Footwear (trainers)
 & 25 & 9 \\ 
\hline 
\end{tabular} 
\label{tab:leakageRatesProdSF6}
\end{table}
\newpage

Table \ref{tab:leakageRatesProdSF6} points out that most of the emissions into the atmosphere is from high voltage switchgear called GIS. This is due to the leakage rate and the installed capacity of 202 tonnes of SF$_6$.

DISKUSJON
Since most of the users of medium voltage switchgear is not a part of the user group it is possible that the government will force a SF6 free technology or ban the use of SF6 in this range rather than just make it voluntary. ABB produces a lot of equipment that do get exported out of the contry. It is not nesicary god handling of SF6 in these contryes in the same way as in Norway. This might result in futer leaks. Air will in this case be good.

\section{Results}

\newpage

\section{Discussion}
\subsection{The difference between air and SF$_6$ as interrupting medium} \label{sec:airandsf} 
Air and SF$_6$ are both fairly good interruption gasses and have been successfully used in the past to interrupt high currents at high voltages. The primary difference between the two gases are the dimensions of the switchgear. Traditionally, circuit breakers using air as an interrupting medium, and not SF$_6$, have been lager and used higher pressures to break the current. When producing circuit breakers and lager load brake switches optimization and careful design regarding material usage and dimensions most be taken into account. The story is however a bit different when designing load break switches for medium voltage levels. In most cases design principles from circuit breaker designs have been scaled down and reused in load break switches. This makes reason to believe that some of the compact load break switch designs that are on the market today is in fact over scaled. If they are over scaled it might be possible to keep the dimensions equal and exchange the interrupting gas from SF$_6$ to air, however careful optimization must be done to meet the demands to dielectric strength and interruption capabilities. Since most of the research on switchgear technology is done on circuit breakers the difference between air and SF$_6$ on a medium voltage load break switch may not be directly linked with the difference when regarding circuit breakers. However some of the main differences and challenges with the use of air instead of SF$_6$ is pointed out below.

\subsubsection*{Electrical conductivity}
As expressed in section \ref{sec:eleCondArc} the transaction from isolating stage to a conductive stage for both air and SF$_6$ is quit fast, and the conductive properties are good for both gases. In figure \ref{fig:AirandSF6ConComp} below the conductivity of air and SF${_6}$ is compared. Even though air and SF$_6$ have approximately the same conductivity when dissociated there are some differences in the ionization products of the gases. These differences represent one of the biggest differences regarding electrical conductivity.

\begin{figure}[H]
\centering
\includegraphics[scale=0.6]{Bilder/Theory/emissionsSF6Norway.png}
\caption{The conductivity of air and SF${_6}$ \cite{bib:StatSF6}.} \label{fig:AirandSF6ConComp}
\end{figure}

Decomposed SF${_6}$ consist of a high concentration of ionized fluorine, both F$^+$ and F$^{++}$ is present. Since ionized fluorine is one of the most oxidative particles know, and therefore it will attract electrons. The moment right after the arc has extinguished it is a high concentration of free electrons in the contact gap. Many of these will be removed by the ionized fluorine. In air ionized oxygen is the strongest oxidative particle and it oxidative properties are similar to ionized fluorine. However the concentration of ionized oxygen is far lower in an air based switchgear than the concentration of ionized fluorine is in an SF$_6$ based one.

\subsubsection*{Thermal conductivity}
When regarding current interruption and the possibility for re-strike it is the thermal conductivity that differ the most between air and SF$_{6}$. From figure \ref{fig:tempConGas} in section \ref{sec:HeatTransport} this difference is pointed out. For the temperature ranges that can be expected in a typical LBS during the high current stage of the interrupting process air has a fairly high thermal conductivity while SF$_6$ has a low. Giving SF$_6$ a clear advantage over air. The most critical part in an interruption is how the interruption medium behaves the moment right before CZ. It is in the temperature range that occurs in this stage of the interruption the biggest challenge with using air as an interruption medium applies.

When the current amplitude is rising or is high the temperature in the plasma channel will be high due to energy dissipation in the arc. At these stage in the interruption process it is preferred to have a medium with a low thermal conductivity. A low thermal conductivity will ensure that the energy is stored inn the arc, therefore increasing its temperature even more rather than dissipating it out to the surrounding. At high temperatures SF$_6$ have a fairly low thermal conductivity while air have a high one. The high thermal conductivity of air in this temperature area will result in a poorer conductivity due to temperature loss which again may result in a slightly higher arcing voltage. The high temperature conductivity of air and the larger energy dissipation, due to higher arcing voltage will also result in heating of the surroundings. The gas surrounding the arc might act as an energy reservoir storing heat. This is making the plasma channel more resistant to fast changes in temperature, resulting in a longer cool-down time for the arc.

Even though the difference in thermal conductivity at high temperatures is a challenge when dealing with air instead of SF$_6$, it is the difference between the heat transport properties at lower temperatures that are the most challenging. At the moment right before CZ it as an advantage that the thermal conductivity of the gas is high. Since the current amplitude is low and falling, and the energy in the plasma channel is transferred out to the surroundings fast, the temperature will drop fast. The speed of the temperature drop will depend on the temperature of the surroundings and the thermal conductivity of the gas. The moment the arc extinguish and the current is zero is is crucial that the ionized gas recombines. The speed of the recombination is mainly dependent on the temperature of the gas. Since SF${_6}$ have a high thermal conductivity for temperatures that will occur around CZ the gas will cool down quickly and recombine fast. Air will however use much longer time than SF${_6}$ to cool down and recombine, due to its low thermal conductivity at this stage in the interruption process. This means that in the moment right after CZ there will be a lot of ionized air particles between the contact plates, but in an SF${_6}$ based breaker the gas will mostly have recombined. This property will result in a higher chance of thermal re-ignition in an air based breaker, and set stronger demands to the cooling mechanism.

\subsubsection*{Dielectric properties} %kanskje slå sammen med kap. nedenfor
Even though this research project is mainly focused on thermal re-strikes in a LBS a few things should be mention about the difference in dielectric properties between SF${_6}$ and air. Figure \ref{fig:breakDownVoltage} indicates the different break down voltages for SF$_6$ and air for a gas filled gap at 1 mm with a homogeneous electrical field. As the figure points out SF$_6$ has a much higher break down strength than air, approximately three times higher, but this depends on the pressure. 

\begin{figure}[H]
\centering
\includegraphics[scale=0.6]{Bilder/Theory/emissionsSF6Norway.png}
\caption{Breakdown voltage at certain pressure levels of different gases in a homogeneous field with a gap space equal 1 mm  \cite{bib:TET4160HVIM}.} \label{fig:breakDownVoltage}
\end{figure}

The huge difference in break down voltage is mainly due to the high electron negative properties of SF$_6$. A electron negative gas are complex molecular structure mostly consisting at atoms from the halogen group in the periodic system, usually chlorine or fluorine \cite{bib:TET4160HVIM}. These atoms can easily capture free electrons because they lack one electron in the outer shell, when capturing a electron they become negatively charged ions. Therefore it is possible to conclude that in an electron negative gas the concentration of free electrons will be low, but due to ionisation of the gas the density of negative ions will be lager. However the weight of the negatively charge gas molecules makes them less mobile than the much lighter free electrons. This makes the impact on break down voltage from free electrons much higher than the impact from ionized molecules.

\subsubsection*{Chemical properties}
-contaminants in SF6 like water, makes flouine acid.
-cold temperatures makes droplets and towphase.
\subsubsection*{Environment}

\newpage

\begin{thebibliography}{10}


\bibitem{bib:HVEbreak} \textit{M. Runde}, \textit{Current Interruption in Power Grids}. Trondheim: Norwegian University of Science and Technology, 2013

\bibitem{bib:SF6PI} \textit{L.G. Christophorou, J. K. Olthoff, and R.J. Van Brunt}, "\textit{Sulfur Hexafluoride and the Electric Power Industry}", \textit{IEEE Electrical Insulation Magazine, vol. 13, No. 5, pp. 20-24}, Oct. 1997.

\bibitem{bib:comSub} \textit{amesimpex.com}, \url{http://www.amesimpex.com/images/unitised_sub_002.jpg}, \textit{26.9.2013}

\bibitem{bib:CIAMVLBS} \textit{E. Jonsson, N. S. Aanensen and M. Runde}, "\textit{Current Interruption in Air for a Medium Voltage Load Break Switch}", \textit{IEEE Trans. Power Delivery}, to be published.

\bibitem{bib:KlimaKur2020} "\textit{KLIMAKUR2020}", Oslo: Klima- og forurensningsdirektoratet, 2010

\bibitem{bib:CBAC} \textit{W. Rieder}, "\textit{Circuit breakers, Physical and engineering problems, III-Arc-medium considerations}", \textit{IEEE spectrum, pp. 80-84}, Sept. 1970.

\bibitem{bib:IPSF6AQM} \textit{W. Hertz, H. Motschmann and H. Wittel}, "\textit{Investigations o the Properties of SF$_6$ as an Arc Quenching Medium}", \textit{Proceedings of The IEEE, vol. 59, NO. 4, pp. 485-492}, April 1971.

\bibitem{bib:TDCIGBB} \textit{W. Hermann}, "\textit{THEORETICAL DESCRIPTION OF THE CURRENT INTERRUPTION IN HV GAS BLAST BREAKERS}", \textit{IEEE Transactions on Power Apparatus and System, vol. PAS-96, NO. 5, pp. 1546-1555}, Sept./ Oct. 1977.

\bibitem{bib:consSF6} \textit{esrl.noaa.gov}, \url{http://www.esrl.noaa.gov/gmd/webdata/iadv/ccgg/graphs/pdfs/ccgg.MLO.sf6.1.none.discrete.all.pdf}, \textit{17.10.2013}

\bibitem{bib:StatSF6} \textit{K. L. Hansen}, "\textit{Emissions from consumption of HFCs, PFCs and SF$_6$ in Norway}", \textit{Statistics Norway/Department of Economic Statistics/Environmental Statistics
}, 2007.

\bibitem{bib:regSF6Miljo} \textit{regjeringen.no}, \url{http://www.regjeringen.no/nb/dep/md/dok/regpubl/stmeld/2011-2012/meld-st-21-2011-2012/5/5.html?id=682932}, \textit{21.10.2013}



\bibitem{bib:TET4160HVIM} \textit{E. Ildstad}, "\textit{High Voltage Insulation Materials}", \textit{Trondheim: Norwegian University of Science and Technology, 2012
}, August 2012.

\end{thebibliography}
\end{document}