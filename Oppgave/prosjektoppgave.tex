\documentclass[10pt,a4paper]{article} %hvor stor skrift?
\usepackage[latin1]{inputenc}
\usepackage[english]{babel}
\usepackage{amsmath}
\usepackage{amsfonts}
\usepackage{amssymb}
\usepackage{makeidx}
\usepackage{graphicx}
\usepackage{hyperref}
\usepackage[left=2cm,right=2cm,top=2cm,bottom=2cm]{geometry}
\usepackage{float}
\usepackage{multirow}

%symbolliste


\usepackage{makeidx}
\makeindex
%symbolliste slutt

\author{Anders Dall'Osso Teigset}
\title{PUFFER MECHANISM SKRIV I STORE BOKSTAVER}

\begin{document}
\maketitle
%SummaryNyttige pdf-filer/SF6conduct.pdf
\newpage
\section*{Summary}
\newpage
\tableofcontents
\newpage
\section{Introduction}
A load break switch should be able to interrupt currents less or equal the maximum load current in a distribution system. When interrupting a current the two contacts which a switch consists of start separating producing a gap between each other. Normally the current is not interrupted by this and an electrical arc ignites and burns in the contact gap \cite{bib:HVEbreak}. The arc consists of plasma, which is a mixture of negative and positive ions as well as electrons. Due to the energy dissipation produced by the arc the temperature in the plasma is very high. The electrical conductivity of the plasma channel is dependent of the temperature produced by the arc. When a high current is flowing it is almost a perfect conductor, but at low temperatures the conductivity is much poorer.

An AC-current have a natural zero crossing, as the current approaches this point the arc will start cooling and extinguish if properly cooled at this point of the power cycle. The working principle of a switchgear is to cool the arc sufficiently when the current approaches zero and then quench the arc when the current is zero. At this moment the current is interrupted and a voltage builds up across the open contacts. This voltage is called the recovery voltage. The steepness and amplitude of the recovery voltage \cite{bib:HVEbreak} and the design of the switchgear will decide if a new arc ignites between the contacts after the current zero. If an arc re-ignites ether by a thermal or a dielectric breakdown the interruption has failed. 

Upon till now SF$_6$ and vacuum based technology have been dominating the compact medium-voltage switchgear market. Air insulated switchgear does exists but they are space consuming and are not applicable for use in a compact substation design. A compact substation is one of the most common designs for substations in the medium-voltage level of the distribution system. Figure \ref{fig:compact substation} displays a compact substation which can be used in the medium-voltage distribution system. The switchgear is a module that can be detached from the compact substation and removed trough its front panel. As figure \ref{fig:compact substation} indicates there are a limited amount of space available for the switchgear. Therefore the main challenge for an air insulated switchgear design for this kind of application is set by the amount of space available in the compact substation.

\begin{figure} [h]
\centering
\includegraphics[scale=0.5]{Bilder/Introduction/general_substation.jpg}
\caption{Compact substation with open front panel \cite{bib:comSub}} \label{fig:compact substation}
\end{figure}

The main choice of interrupting media in switchgear has been SF$_6$ since its discovery in the 1970ies. The gas exhibits many properties which is well suited for an isolation gas. It is highly electron negative which gives it good dielectric strength and arc-interrupting capabilities. The breakdown voltage of SF$_6$ is almost three times higher than air at atmospheric pressure \cite{bib:SF6PI}. It has excellent heat transfer properties as an interrupting medium \cite{bib:SF6PI}. The gas also reforms itself when dissociated under the high temperature conditions in an electrical arc. SF$_6$ produces no polymerization, carbon or other conductive deposits during arcing. It is also chemically compatible with most insulation materials and conductive materials \cite{bib:SF6PI}. The gas is also well suited for use in low temperature environments since its boiling point is fairly low even at high pressures. The gas in its stable form is nontoxic, non-flammable and non-explosive. It is also thermally stable and do not decompose at normal operating temperatures for a closed switch \cite{bib:SF6PI}. SF$_6$ based switchgear tends to be cheap to produce relative to other designs, and the gas itself is also affordable at today prices.


SF$_6$ have some disadvantages, when exposed to electrical discharge or arcing it forms highly toxic and corrosive compounds \cite{bib:SF6PI}. It is also hard to remove non-polar contaminants like air and its breakdown voltage is sensitive to water vapour and conductive particles. However it biggest downside is probably that it is an effective infrared absorber. This makes it a strong greenhouse gas \cite{bib:SF6PI} and it is regarded to be almost $20000$ times as potent as CO$_2$.

0.4 \% of the greenhouse gas emissions in Norway is due to SF$_6$ and is mainly used in switchgear or other high voltage equipment \cite{bib:KlimaKur2020}. In Norway the use of SF$_6$ is regulated through a voluntary contract between the environmental department and the end user, mainly the power companies \cite{bib:KlimaKur2020}. If a compatible air based switchgear design is released on the market it can be assumed that the power companies will be interested in the new technology. It is also possible that the government will restrict the use of SF$_6$ gas if the private sector do not follow the guidelines of the environmental department.

A test switch have been developed which allows adjustment of many design parameter such as nozzle and contact geometry, contact movement and gas pressure. The contacts consist of one female contact, or tulip and one male contact, or pin. The tulip is immovable, while the pin can be opened by a spring trigged by an electromagnet. Previous results have illustrate that most of the successful interruptions have occurred when the pin is outside the nozzle \cite{bib:CIAMVLBS}. Cold air simulations done by Nina Sasaki Aanensen of the different geometries have stated that volumetric flow of air is much lower when the pin is inside the nozzle. The results from previous testing and cold air simulations might suggest that the pin is clogging the nozzle and preventing a good air flow. Therefore a new nozzle geometry has been designed with a doughnut area, the area between the nozzle and the pin, which is bigger than the area of the pin. This is assumed to give a volumetric flow of air that is almost the same when the pin is inside the nozzle as well as outside. The theory has been verified by cold air simulations. However the clogging effect generated by an arc has not been taken into account and it is expected that this will have an impact on the volumetric flow. Two nozzles have been constructed of the new nozzle geometry one with an doughnut area of $44 \ mm^2$ and the other one with an area of $66 \ mm^2$. It is expected that the speed of the air will be constant and the volumetric flow will increase with the doughnut area. This will give an opportunity to test which of these parameters that influence the current interruption capabilities the most.

Several different nozzle geometries are going to be tested and it is assumed that the chance for a successful current interruption inside the nozzle will be approximately the same as outside the nozzle. Since the size of the arc compared to the pin will vary between the different nozzle geometries and current amplitudes, a clogging effect is assumed to occur and might be revealed as an lower interruption success rate for certain geometries.

\newpage


\newpage
\section{Theory}
\subsection*{Symbol list}
\begin{description}
\item{$U_{recovery}$} \ -- \ Voltage that appears across the terminals after current interruption.
\item{$U_{left}$} \ -- \ Voltage on the left hand side of the switch after current interruption.
\item{$U_{right}$} \ -- \ Voltage on the right hand side of the switch after current interruption.
\end{description}
\newpage
\subsection{Typical switchgear design and interruption sequence} \label{sec:genDes}
Most of the information in section \ref{sec:genDes} is collected from \textit{"Current Interruption in Power Grids"} by Magne Runde \cite{bib:HVEbreak} \newline

\subsubsection{Switchgear design and operation} \label{sec:InterruptCurrent}
Switchgear can be divided into four main categories:
\begin{itemize}
\item Disconnector Switch
\item Load Break Switch
\item Circuit Breaker
\item Earthing Switch
\end{itemize}
This report will focus on the load break switch (LBS) design. A LBS is a switchgear design to be able to interrupt currents with magnitude that is equal or less than the rated maximum continuous current in a transmission system. A LBS must fulfil the following demands to meet the requirements of the application area:

\begin{itemize}
\item When closed:
	\begin{itemize}
		\item It must act as an perfect conductor.
		\item Be capable to interrupt any currents that may arise, without generating too high over-voltages. 
	\end{itemize}
\item When open:
	\begin{itemize}
		\item It must be an perfect isolator.
		\item Be able to close without welding the contacts together, event under short-circuit conditions.
	\end{itemize}
\end{itemize}

A typical operation sequence for a switch is as follows. First a control signal enters the switch and activates the opening mechanisms. The contacts starts to open and a gap forms between them, at the same time an electrical arc ignites between the contacts, burning in the gap. The gap is filled with some kind of interrupting medium which is usually a gas. For a altering current with a frequency of 50 Hz the current will cross zero 100 times in the second, this crossing is called the current zero (CZ). Direct current interruptions will not be explained in this report. 

At the CZ the arc will extinguish because the current is zero, and a voltage will build up between the contacts. This voltage is called the recovery voltage and is defined in equation \eqref{eq:U_rec}. Where $u_{left}$ is the voltage on the left side and $U_{right}$ is the voltage on the right side of the open switch. If the recovery voltage is increasing to	o fast or have a high amplitude a re-ignition of the arc may occur. There is two different kinds of re-ignition: thermal and dielectric re-ignition. Thermal re-ignition take place right after CZ and is mainly dependent on the recovery voltage's steepness. As the recovery voltage rises and a thermal re-ignition is avoided a dielectric re-ignition may occur. This kind of re-ignition is largely dependent on the amplitude of the recovery voltage. The chance for re-ignition will not only be determinated by the recovery voltage, but also depend what kind of interrupting medium that is being used. Other factors like contact material, geometry, and cooling mechanisms are also important.

\begin{equation} \label{eq:U_rec}
u_{recovery}=u_{left}-u_{right} \cite{bib:HVEbreak}
\end{equation} 

Plasma is generated when gas or metal vapour is heated to very high temperatures, at a certain point the molecules in the gas decomposes to ions and free electrons. This mixture is called plasma, and makes up most of the components in an electrical arc. This report describe some of the properties of an arc burning in atmospheric pressure or above. Arcs in vacuum behaves different, and will not be featured in this report.


In a LBS it is important to assure that the switch in closed position acts as a perfect conductor with a low contact resistance. Copper or aluminium is ideal materials and are often used to ensure that this aspect of the contact is met. Sometimes the contact surface is plated with tin, gold, silver or platinum to ensure a low contact resistance between the contact plates. The main problem with electrical losses in the switch is heat generation which may speed up metal creeping processes and other ageing related processes in the switch. Contact plates made out of aluminium are especially vulnerable to creeping.
 
To withstand the harsh conditions that occurs when an arc burns between the contacts the contact material has to meet strict requirements. It has to tolerate high temperatures, arc erosion, welding and other stresses that may apply when closing or opening an energized contact. Aluminium and copper is not fit for these tasks since they will melt or erode from stresses of an arc. It is common to use composites of metals with good electrical conduction with heat resistant oxides. For high current and voltage switches it is usual to use a composite of silver or copper together with tungsten or tungsten carbide. These materials are very resistant against heat, erosion and welding, but they have a high resistance. Therefore it is common on breakers above $70 \ kV$ to have two sets of contacts, one arcing contact and one main contact. The main contact is the first contact to open and the last one to close. This is to ensure that an arc do not start to burn between the contact, which makes it possible to use aluminium or copper as contact materials. The arcing contact is the last contact to open, and the first one to close. This will ensure that the arc burns between the arcing contact and not between the main contact. This contact arrangement is also common in the compact LBS systems available today, and many other common switchgear designs.

\subsubsection{The puffer principle}
To quench the arc, several mechanisms and interrupting mediums can be applied. In today's compact LBS systems based on SF$_6$ the puffer mechanisms is commonly used. A brief description of current interruption is explained in section \ref{sec:InterruptCurrent}. To obtain a successful interruption of an arc it is possible to cool the arc and interrupting medium down, as well as blow away charged particles and vaporised metal between the contact points. This is the main purpose with the puffer mechanism. 

As SF$_6$ entered the industry in the 1960s it was in the form of dual pressure breakers. The SF$_6$ based dual pressure breakers had one high pressure chamber, and one low pressure chamber. A valve from a high pressure chamber opens during opening operations, generating a high speed SF$_6$ blast, guided by a nozzle to hit the arc burning between the contacts in the low pressure chamber. This design have two major disadvantages when using SF$_6$. The switch requires heating to maintain the pressure in the high pressure chamber to avoid condensation of the SF$_6$ gas. It also uses a compressor to pump the gas from the low pressure chamber to the high pressure chamber. This adds to the complexity to the switchgear and may result in more maintenance. The double pressure design was replaced with the newer single pressure design. The single pressure design has only one pressure chamber with low pressure except during interruptions when the chamber it self becomes a high pressure chamber.

The single pressure design uses a puffer or the self-blast mechanism to quench the arc, or in some cases a combination of both. Common for both puffer and self-blast mechanisms are that they do not use a compressor to generate the gas flow, but uses the energy stored in the switching mechanisms or generated from the blast itself to interrupt the arc. These mechanisms works the same way for both circuit breakers and load break switches, but will in a load break switch be smaller and less complex. This is because the current amplitudes are smaller and therefore a lower pressure is needed to obtain a successful interruption.

The puffer mechanism is based on a piston to generate a flow of gas in the switchgear to extinguish the arc. A typical design of this system is by using a fixed piston integrated as a part of the contact design. A gas reservoir trapped between the arcing contact and the piston is pushed out as the contact moves apart form each other. Figure \ref{fig:CircutBreakPuff1} displays a typical interruption sequence of breaker based on the puffer design.


\begin{figure} [H]
\centering
\includegraphics[scale=0.8]{Bilder/Theory/CircutBreakPuff1.png}
\caption{Interruption sequence in a puffer breaker \cite{bib:HVEbreak}} \label{fig:CircutBreakPuff1}
\end{figure}

When the breaker is closed as seen in figure \ref{fig:CircutBreakPuff1}a there is a gas volume \textit{(V)} trapped between the piston \textit{(P)} and the arcing contact \textit{(8) and (6)}. At the moment the movable part of the arcing contact \textit{(8)} is pulled down the volume decreases because of the fixed piston and an increase in pressure due to compression of the gas occurs. Figure \ref{fig:CircutBreakPuff1}b illustrate that the main contact is open, and that the current now only flows through the arcing contact.

The next stage of the interruption sequence is pointed out in figure \ref{fig:CircutBreakPuff1}c. The arcing contacts have now separated and an arc \textit{(A)} has ignited between the contacts. The pressurised gas that previously was trapped between the piston and the arcing contact is now released. The gas flow is guided by a nozzle \textit{(9)} that is fixed to the movable arcing contact. The gas flow will cool down the arc and blow away charge carriers between the contact plates. If a sufficient gas flow is obtained the arc will not re-ignite after current zero and neither extinguish before current zero, so that current chopping is avoided.

The gas flow is partially dependent on the cross-section of the arc, which again is dependent on the current amplitude. A large current resulting in an large arc may block the hole in the nozzle preventing a gas flow. This is called current clogging and may occur for certain nozzle designs at high current interruptions. In such an event the pressure in the gas reservoir will increase further due to compression from mechanical moment of the arcing contact and thermal expansion in the gas because of heating from the arc. When the current amplitude approaches zero its cross-section will decrease and the clogging effect will end. This will result in a powerful gas blast onto the arc, as indicated in figure \ref{fig:CircutBreakPuff1}d. For smaller current amplitudes the arc cross-section is smaller and a blocking effect does not occur in the same extent. This generates a less intense gas flow, preventing current chopping.
 
\begin{figure} [H]
\centering
\includegraphics[scale=0.4]{Bilder/Theory/selfBlast.png}
\caption{Expulsion chamber in a breaker using the self-blast mechanism \cite{bib:CBAC}} \label{fig:selfBlast}
\end{figure}

The self-blast, or third generation breaker, was developed with the goal of reducing mechanical power of the operating system, making it cheaper and less complex. Figure \ref{fig:selfBlast} illustrates the working principle of a breaker using self-blast to interrupt an arc. The difference between self-blast and puffer mechanism is that the puffer mechanism increases the pressure by reducing the volume. Rather than the self-blast design which have a constant volume relying on a raise in temperature to increase the gas pressure \cite{bib:CBAC}. The self-blast design uses the heat generated from an arc burning between the arcing contacts to interrupt the current. The gas expands as it is heated by the burning arc, this increase in pressure leads to a gas flow on the arc, which cools it down leading to the quenching of it.

There are some disadvantages with the self-blast principle when compared to the puffer mechanism. The self-blast have a lower dielectric strength due to hot gas between the contacts after CZ. This gives a higher chance of re-ignition since hot gas has a lower ionisation energy than cold gas. The design also not well suited to break smaller currents. This is because the arc is less intense and therefore do not heat the gas sufficiently to create a strong enough blast. Therefore it is common to combine self-blast and puffer mechanism in a hybrid design, so that it can handle both small and large currents. A compact LBS design using air as interrupting medium will probably rely on a puffer design. This is because of the small current an LBS is usually facing compared to a circuit breaker.

\subsection{Air flow considerations} \label{sec:AirFlow}
\textit{ Most of the information in section \ref{sec:AirFlow} is collected from the article "Air Flow Investigation for a Medium Voltage Load Break Switch" by N. S. Aanensen, E. Jonsson, and M. Runde} \newline

From Bernoulli's equation it can be deduced that the maximum velocity of the gas flow is obtained from a set pressure, \textit{p}, difference as given in equation \eqref{eq:Bernoulli}.

\begin{equation} \label{eq:Bernoulli}
v_{max}=\sqrt{2 \Delta \frac{p}{\rho}}
\end{equation}

$\rho$ is the mass density of the fluid. The fluid is assumed to have an ideal flow without viscosity and incompressible behaviour. $v_{max}$ will occur on the narrowest part of the geometry. The velocity in the other parts of the nozzle is defined by a constant volumetric flow rate. The relation between volumetric flow rate \textit{Q} and $v_{max}$ is presented in equation \eqref{eq:flowRate}.

\begin{equation} \label{eq:flowRate}
Q=v_{max} A_{contact}
\end{equation} 

If table \ref{tab:contGeoPara} is consulted it indicates that the area $A_{contact}$ is smaller than both $A_{ring}$ and $A_{nozzle}$ for both geometries. This means that for the geometries in this experiment $v_{max}$ is allays set by the area of the female contact and the gas velocity will be the same whenever the pin is inside or outside the nozzle because the volumetric flow is constant. The velocity of the gas flow in the nozzle is set by equation \eqref{eq:VolumetricFlow}.
\begin{equation*}
Q=v_{max} A_{contact} = v_{nozzle} A_{nozzle}
\end{equation*}


\begin{equation} \label{eq:VolumetricFlow}
v_{nozzle}= v_{max}\frac{A_{contact}}{A_{nozzle} }= v_{max} (\frac{d}{D})^2
\end{equation} 

$v_{nozzle}$ is approximately the same for geometry \textit{a} and \textit{b} since the fraction $\frac{D}{d}$ form table \ref{tab:contGeoPara} is almost the same for both geometries. The volumetric flow rate will increase with the increasing geometry size according to A$_{contact}$

The deduction above is based on the assumption that the fluid is non-compressible and have a ideal flow without viscosity. This is probably not valid since turbulence, wall effects, variations in density and temperature are likely to be present. For pressure differences above 0.7 bar it is expected that sonic speeds can be reached by the gas flow. Simulations of compressible gas flows at ambient temperature have therefore been carried out by Nina Sasaki Aanensen, to verify that the two geometries in fact will have similar air flow velocity for equal upstream pressure and that the air flow velocity is approximately equal when the pin is inside and outside the nozzle. 

It is safe to assume that an burning arc between the contacts will have an impact on the air flow, as it heats both the air and surroundings. This has not been accounted for in the cold flow simulations and therefore these are not expected to be accurate. It is difficult to know exactly how an arc interacts with the gas flow and a good simulation tool for load currents is still to be obtained.

%\subsubsection{Test switch}
%A new medium voltage laboratory for load break switch development have been designed with the possibilities to vary important design parameters for a load break switch. When the industry develops switchgear technology they tend to alter several different parameters at the same time. For instance, if they wish to alter the speed of the airflow in a puffer LBS, they might increases the separation speed of the contacts, since these functions often are linked in a puffer design. This will also result in a higher pressure not only and not only a higher volumetric flow. This will make it hard to conclude which parameter that was the most critical for a successful interruption. Therefore a laboratory switch have been design where each of the different parameters can be adjusted without changing other critical parameter. This will make systematic researching possible and optimisation of certain designs criteria, like dimensions, will be a lot simpler.

%\begin{figure} [h]
%\centering
%\includegraphics[scale=0.5]{Bilder/Theory/SchematicTestSwitch.png}
%\caption{Test switch} \label{fig:testSwitch}
%\end{figure}

%Figure \ref{fig:testSwitch} shows the physical appearance of the switch and how the different parameter can be adjusted. BLA BLA TAKE A PICTURE OF THE SWITCH, WRITE SOME NUMBERS ON IT. TAKE ABOUT THEM AND WITCH PARAMETERS THAT CAN BE ADJUSTED.

%\begin{figure} [h]
%\centering
%\includegraphics[scale=0.5]{Bilder/Theory/SchematicTestSwitch.png}
%\caption{Tulip, nozzle and pin, L and D is the length and inner diameter of the nozzle, x is contact position. Dimensions are in millimetres.   \cite{bib:CIAMVLBS}} \label{fig:SchematicTestSwitch}
%\end{figure}

%Figure \ref{fig:SchematicTestSwitch} display the former nozzle geometries and it can be clearly seen from this picture that the doughnut area is smaller than the tulip area. It is possible that this will give a clogging effect when the pin is inside the nozzle and result in a poorer airflow and interrupting capabilities for this situation. In figure \ref{fig:differentGeometries} the different contact geometries that have been made is shown. Geometries a1, b1, and c1 have been tested and the results showed that most of the successful interruptions happen when the pin was outside the nozzle. It is possible that this is the result of a clogging effect that takes place between the nozzle and pin. Therefore geometries b2 and c2 was created. The geometries is similar to b1 and c1 since they have the same nozzle area, respectively $44 \ mm^2$ and $66 \ mm^2$. The volumetric airflow will however be smaller for b2 and c2 because it have been created a choking point in the tulip. 

%\begin{figure} [h]
%\centering
%\includegraphics[scale=0.6]{Bilder/Theory/differentGeometries.png}
%\caption{Different contact geometries \cite{bib:CIAMVLBS}} \label{fig:differentGeometries}
%\end{figure}

%It is expected that b2 and c2 will need a higher pressure than b1 and c1 to perform a successful interruption, due to the lower airflow. However it is expected that they will have an equal successrate no matter if the pin is inside the nozzle or outside the nozzle.

\newpage
\subsection{Interrupting currents} 
%\subsubsection{Phenomenological description of a typical current interruption} \label{sec:InterruptCurrent}
%A interrupting sequence in an AC power system follows a typical pattern. During normal service the contacts are closed and current is flowing through the switchgear without electrical losses. When a command signal to interrupt the current is sent to the breakers driving mechanism the contacts are set to motion. A gap opens between them and is filled with an interrupting medium. An arc usually forms between the contacts and the current continues to flow trough the breaker. This arc mainly consists of plasma which is composed of electrons and other charge carriers like positive and negative ions. Due to energy dissipation in the arc the temperature in the plasma is generally high, but strongly dependent on the amplitude of the electrical current passing through the breaker \cite{bib:HVEbreak}. The system load is the main factor when assessing the electrical current flow through the breaker, the effect from the arcing voltage is often disregarded. Because of the limited time span when assessing breaking operations the sinus curve of the current at a given time is needed to be taken account for. Plasma with an high temperature tends to have a higher electric conductivity than plasma with an lower temperature. Electrical conductivity in an plasma channel is also strongly depend on the medium the plasma is burning in, the difference between air and SF${_6}$ will be discussed in section \ref{sec:airandsf}.

%The arcing voltage, defined as the voltage drop over the arc, is dependent on the temperature and the cross-section of the arc \cite{bib:HVEbreak}, since these are depending on the current the arcing voltage can be regarded as constant when the current exceeds a certain level. This is further explained in section \ref{sec:elarcs} . When the current approaches its zero crossing (CZ) the energy dissipation in the arc decreases and the plasma cools down. With different cooling mechanisms as mention in section \ref{sec:genDes} the breaker cools down the arc. If done sufficient the electric conductivity of the interrupting medium becomes low, and it behaves as an isolator. This will quench the arc as the current amplitude reaches zero.

%The moment after the arc have been quenched a voltage builds up over the contacts, this is called the recovery voltage \cite{bib:HVEbreak}. If a successful interruption is to occur, a re-ignition of the arc must be avoided as the recovery voltage increases. The probability of an re-ignition is set by the steepness and amplitude of the recovery voltage \cite{bib:HVEbreak}. There is two different kinds of re-ignition: thermal and dielectric re-ignition. Thermal re-ignition take place right after CZ and is mainly dependent on the recovery voltage's steepness. As the recovery voltage rises and a thermal re-ignition is avoided a dielectric re-ignition may occur. This kind of re-ignition is largely dependent on the amplitude of the recovery voltage. The recovery voltage can be mathematically expressed as described in equation \eqref{eq:U_rec}. This formula presents a opportunity to analyse the steepness and amplitude of different interrupting scenarios mathematically.

%\begin{equation} \label{eq:U_rec}
%u_{recovery}=u_{left}-u_{right} \cite{bib:HVEbreak}
%\end{equation} 

%Where $u_{left}$ is the voltage on the left hand side of the breaker and $u_{right}$ is the voltage on the right hand side.

\subsubsection{Voltage drop in a static electric arc} \label{sec:elarcs}

When working with arcs it is common to differentiate between dynamic and static arcs. The static arc is established during switching operations in a DC power system after the transient phenomenas have faded out. An dynamic arc is an arc that occurs in AC power systems and during the transient part of a DC interruption, or when the cooling of the arc varies with time, like in most puffer designs. Even though all arcs in an AC system is to be regarded as dynamic arcs, it is possible to regard them as static during a short period of time \cite{bib:HVEbreak}.

Figure \ref{fig:staticArcChar} displays the static arc characteristic for an arc burning in a gas filled gap. The figure is only valid for pressure levels equal or above atmospheric pressure. The static arc characteristic gives a relation between the arcing voltage and the electrical current flowing in the gap. The values on the axis is only approximations and varies with the gas type and contact material. The length of the gap is also an important factor the magnitude of the arcing voltage. 

\begin{figure}[H]
\centering
\includegraphics[scale=1]{Bilder/Theory/staticArcChar.png}
\caption{Static arc characteristic  \cite{bib:HVEbreak}.} \label{fig:staticArcChar}
\end{figure}

At low currents the arcing voltage is high, but decreases rapidly with increasing currents. For higher currents the voltage drop is approximately constant, until the current reaches a certain level and the arcing voltage starts to increase. The difference between air and SF$_6$ as an interrupting medium will be discussed further in section \ref{sec:airandsf}, but on a general basis it is likely to assume that SF${_6}$ have a lower arcing voltage if compared to air for a certain breaking operation. This implies that the energy dissipation in air will be higher, especially when breaking smaller currents where the arcing voltage difference is greatest. This is highly relevant for an LBS since most of the switching duties occurs at smaller currents.

An static electrical arc might be regarded as divided into three regions \cite{bib:HVEbreak}.
\begin{description}
\item[Chatode region] The voltage drop, $V_c$, is usually around 20 V.
\item[Arc column]	There is a constant electric field in this region, typical 10 V/cm.
\item[Anode region] The voltage drop, $V_a$, is usually around 3 V.
\end{description}

This relationship is illustrated in figure \ref{fig:potDisArc}.
\begin{figure}[H]
\centering
\includegraphics[scale=0.8]{Bilder/Theory/potentialDistArc.png}
\caption{Cross-section of a stationary arc and the corresponding potential distribution \cite{bib:HVEbreak}.} \label{fig:potDisArc}
\end{figure}

For short arcs the voltage drop will mostly occur close to the cathode, and some what at the anode. In longer arcs more of the voltage drop will occur in the arc column itself.

\subsubsection{Electrical conductivity in an arc} \label{sec:eleCondArc}
Gasses have the ability to be perfect isolators as well as good conductors, mainly depending on the gas temperature. This is due to charged particles and electrons created by dissociation of the molecules in the gas. Air is a mixture of several gasses but might be simplified to consist mostly of nitrogen (N$_2$). In figure \ref{fig:condAir} the conductivity of air can be observed.   

\begin{figure}[H]
\centering
\includegraphics[scale=0.8]{Bilder/Theory/airConduct.png}
\caption{Electrical conductivity of air at atmospheric pressure \cite{bib:HVEbreak}.} \label{fig:condAir}
\end{figure}

The steep increase in conductivity can mainly be explained by the dissociation process and ionization of N$_2$ due to temperature increase. The particle density of nitrogen as it dissociates due to high temperature in the gas is illustrated in figure \ref{fig:Ndensi}. When figure \ref{fig:Ndensi} is compared to figure \ref{fig:condAir} a connection between temperature and the rapid decline of N$_2$, generation of the positive ion N$^+$, and the steep increase in conductivity of air is clearly presented.

\begin{figure}[H]
\centering
\includegraphics[scale=0.8]{Bilder/Theory/particleDensNit.png}
\caption{Particle density for different dissociation products of nitrogen as a function of temperature \cite{bib:HVEbreak}.} \label{fig:Ndensi}
\end{figure}

From figure \ref{fig:Ndensi} the electron positive effect of N$_2$ is also indicated via the generation of N$_{2}^{+}$ molecules. From table \ref{tab:thermalIonisation} the thermal ionisation energy for some gases are presented. This points out that N$_2$ have a significant lower ionisation energy than SF$_6$, and it gives away electrons more easily. However sulphate and fluorine have a much lower ionisation energy, which is products of the dissociation of SF$_6$. This indicates that when SF$_6$ first is dissociated the ionisation and conductivity of the gas rapidly increases. This is also pointed out in figure \ref{fig:SF6densi} which indicates the particle density of SF$_6$.

\begin{table}[H]
\center
\caption{Thermal ionisation energy for some gases \cite{bib:HVEbreak}.}
\begin{tabular}{|l|c|c|}
\hline 
Particle type & Single ionisation [eV] & Double ionisation [eV] \\ 
\hline 
Air & 16.3 &  \\ 
\hline 
N$_2$ & 15.8 &  \\ 
\hline 
N & 14.5 & 44.1 \\ 
\hline 
O$_2$ & 12.5 &  \\ 
\hline 
SF$_6$ & 19.3 &  \\ 
\hline 
S & 10.4 & 33.8 \\ 
\hline 
F & 17.4 &  \\ 
\hline 
\end{tabular} 
\label{tab:thermalIonisation}
\end{table}

\begin{figure}[H]
\centering
\includegraphics[scale=0.5]{Bilder/Theory/particleDensSF6.png}
\caption{Particle density for different dissociation products of SF$_6$ as a function of temperature \cite{bib:IPSF6AQM}.} \label{fig:SF6densi}
\end{figure}

\begin{figure}[H]
\centering
\includegraphics[scale=0.5]{Bilder/Theory/SF6Conduct.png}
\caption{Electrical conductivity of SF$_6$ at atmospheric pressure \cite{bib:IPSF6AQM}.} \label{fig:condSF6}
\end{figure}

In figure \ref{fig:condSF6} the electrical conductivity of SF$_6$ as a function of temperature is presented. At high temperatures, when SF$_6$ are fully ionized, the conductivity is high, almost in the same range as metals. The transaction between the isolating and the conducting stage is quick. 

Decomposed SF$_6$ consists of a high concentration of ionized fluorine, both F$^{+}$ and F$^{++}$. These particles are highly oxidative, which means that they will attract electrons. In air oxygen have this effect, but the concentration of ionized oxygen is far lower than fluorine. During a thermal re-ignition it is the steepness of the recovery voltage that is the most important factor when analysing a thermal re-ignition \cite{bib:HVEbreak}. This is due to the fast acceleration of charge carriers that occurs when the strength of the electrical field between the contacts raises fast, which results in an increase of ionized particles due to collisions. 
   
\subsubsection{Heat transport in an arc} \label{sec:HeatTransport}
There are several different conductive mechanisms in an electrical arc. The effects of these mechanisms vary with temperature, and therefore the heat transport in the arc is strongly dependent upon the temperature. In figure \ref{fig:tempConGas} several common interrupting gases thermal heat conductivity is compared to each other as a function of temperature.

\begin{figure}[H]
\centering
\includegraphics[scale=0.8]{Bilder/Theory/thermalCond.png}
\caption{Thermal conductivity as a function of temperature \cite{bib:HVEbreak}.} \label{fig:tempConGas}
\end{figure}

As illustrated in figure \ref{fig:tempConGas} the thermal conductivity of air differ quite much from the one of SF$_6$. Due to the nature of different stages in current interruption it is desired to use a gas that have a thermal conductivity that suites the different stages right. 

When the current amplitude is rising or is high it is preferred that the thermal conductivity is low. This means that the plasma channel do not heat its surrounding but mainly keep the dissipated energy stored in it self. This will result in a temperature rise in the plasma channel, and a relatively small increase in the surroundings. As explained in section \ref{sec:eleCondArc} a high arc temperature will result in high conductivity in the arc, which gives a low arcing voltage. If the thermal conductivity is high in this region, heating of the surrounding system will occur. This should be avoided since it results in unnecessary dissociation of additional interrupting medium. This might result in a slower transaction between the conductive and isolation stage of the interrupting medium due to the stored energy in the medium and the surroundings, resulting in a higher chance of re-ignition.

At the moment right before CZ it is an advantage that the thermal conductivity of the gas is high. This will result in a fast cool-down time of the plasma channel since both the current amplitude is decreasing and the energy stored in the arc now is permitted out in the surroundings. A gas with high thermal conductivity in this stage of the interruption process will be able to recombine from a ionized and highly conductive to a non-conductive state fast, making it harder for a thermal re-ignition to occur. This is because of the quick cooling of the plasma channel. In gases where the thermal conductivity is low the cooling mechanisms is of great importance since a quick recombination of ionized gas do not occur in the same manner as when the medium is quickly cooled. Therefore removal of hot gas and charge carriers must be done differently, this is described in detail in section \ref{sec:genDes}.

\begin{figure}[H]
\centering
\includegraphics[scale=0.3]{Bilder/Theory/tempZonesArc.png}
\caption{Radial temperature distribution in a plasma channel \cite{bib:TDCIGBB}.} \label{fig:tempDist1}
\end{figure}

The temperature distribution in an plasma channel can be divided into three sections \cite{bib:TDCIGBB} as illustrated with figure \ref{fig:tempDist1}. Zone 1 is the highly conductive arc core and also the zone with the highest temperature. Zone 2 acts as an energy buffer during the decay of the arc while zone 3 is the cold gas surrounding the arc. When using cooling-mechanisms to quench the arc it is primary the second zone of the temperature profile that is cooled. The first zone's temperature will mainly be dependent on the current passing through the arc and will not be influenced by the cooling mechanism is the same degree. If the cooling is sufficient the energy stored in zone 2 when the arc approaches CZ is low and therefore its effect as a energy buffer is reduced, resulting in a rapid decline in temperature in the arc core as the current approaches zero. This makes the interrupting mediums ability to transport energy important when investigating efficient cooling methods. As figure \ref{fig:tempConGas} have pointed out SF$_6$ have the ability to transfer heat between zone 1 and 2 fast in the right temperature range compared to the interrupting sequence, air have a poorer ability to do this.

\begin{figure}[H]
\centering
\includegraphics[scale=0.8]{Bilder/Theory/plasmaChannel1.png}
\caption{Radial temperature distribution in a plasma channel \cite{bib:HVEbreak}.} \label{fig:tempDist2}
\end{figure}

In figure \ref{fig:tempDist2} it is illustrated how the temperature distribution varies with the electrical current. Due to radiation losses in the arc the temperature have a upper limit to about 20 000 K to 30 000 K, at this point the cross-section of the arc will increase rather than the temperature \cite{bib:HVEbreak}. However it is not common for a LBS to experience these temperature ranges and its temperature distribution will mainly be in the lower current part of the figure.

\subsubsection{The difference between air and SF$_6$ as interrupting medium} \label{sec:airandsf}

Air and SF$_6$ are both fairly good interruption gasses and have been successfully used in the past to interrupt high currents at high voltages. The primary difference between the two gases when used as interrupting medium is the required dimensions of the switchgear. Traditionally, circuit breakers using air as an interrupting medium and not SF$_6$ have been lager and used higher pressures to interrupt the current. When producing circuit breakers and lager load brake switches optimization and careful design regarding material usage and dimensions must be taken into account. When designing load break switches for medium voltage levels, it is common in some cases to take design principles from circuit breakers, and have them scaled down and reused in load break switches. This makes reason to believe that some of the compact load break switch designs that are on the market today are in fact over scaled. If they are over scaled it might be possible to keep the dimensions equal and exchange the interrupting gas from SF$_6$ to air. However, careful optimization must be done to meet the demands to dielectric strength and interruption capabilities. Since most of the research on switchgear technology is done on circuit breakers, the difference between air and SF$_6$ on a medium voltage load break switch may not be directly linked with the difference when regarding circuit breakers. However, some of the main differences and challenges with the use of air instead of SF$_6$ is pointed out below.

\subsubsection*{Electrical conductivity}
%As expressed in section \ref{sec:eleCondArc} the transaction from isolating stage to a conductive stage for both air and SF$_6$ is quit fast, and the conductive properties are good for both gases. In figure \ref{fig:AirandSF6ConComp} below the conductivity of air and SF${_6}$ is compared. Even though air and SF$_6$ have approximately the same conductivity when dissociated there are some differences in the ionization products of the gases. These differences represent one of the biggest differences regarding electrical conductivity.

\begin{figure}[H]
\centering
\includegraphics[scale=1]{Bilder/Discussion/conductSF6AndAIR.png}
\caption{The conductivity of air and SF${_6}$ \cite{bib:THFD}.} \label{fig:AirandSF6ConComp}
\end{figure}

If the conductivity of air and SF$_6$ is compared, as in figure \ref{fig:AirandSF6ConComp}, it is possible to observe several similarities in the conductivity when compared to different temperatures. At high temperatures, when both gases are fully ionized, the conductivity is high, almost in the same range as metals. The transaction between the isolating and the conducting stage is fast. If the particle density displayed in figure \ref{fig:Ndensi} and \ref{fig:SF6densi} of the two gases is taken into account, it is possible to observe that the decomposition of SF$_6$ occurs at a lower temperature than nitrogen. This can indicate that the transaction from non-conductive to conducting take place a bit faster for SF$_6$ than air. In a LBS this might slightly influence the interruption capabilities, but it is not the electrical conductivity that is the main difference and challenge between SF$_6$ or air as an interrupting medium. Both SF$_6$ and air have fairly good electrical conductivity profiles for interrupting currents.

However, as the current magnitude approaches zero and the gases recombine, SF$_6$ have one major advantage that air do not possess. Decomposed SF$_6$ consists of a high concentration of ionized fluorine, both F$^{+}$ and F$^{++}$. These particles are considered to be highly oxidative, which means they will attract electrons. In the moment right after CZ there are a lot of free electrons in the gap between the contact plates. It is essential to remove these fast to avoid thermal re-ignition of the electrical arc. In an SF$_6$ based switchgear, many of these free electrons are absorbed by the ionized fluorine. In air, oxygen has this effect, but the concentration of ionized oxygen is far lower than fluorine. This is one of the reasons that makes thermal re-ignition a larger problem for an air-based switchgear and not huge in a SF$_6$ based system.

\subsubsection*{Thermal conductivity}
When regarding current interruption and the possibility for re-ignition, it is the thermal conductivity that differs the most between air and SF$_{6}$. From figure \ref{fig:tempConGas} in section \ref{sec:HeatTransport}, this difference is pointed out. For the temperature ranges that can be expected in a typical LBS during the high current stage of the interrupting process air has a fairly high thermal conductivity, while SF$_6$ has a low one, giving SF$_6$ a clear advantage over air. The most critical part in an interruption is how the interruption medium behaves the moment right before CZ. It is in the temperature range that occurs in this stage of the interruption the biggest challenge with using air as an interruption medium applies.

When the current amplitude is rising or is high, the temperature in the plasma channel will be high due to energy dissipation in the arc. At this stage in the interruption process, it is preferred to have an interrupting medium with a low thermal conductivity. A low thermal conductivity will ensure that the energy is stored in the arc, thereby increasing its temperature even more rather than dissipating it out to the surroundings. At high temperatures, SF$_6$ have a fairly low thermal conductivity while air have a high one. The high thermal conductivity of air in this temperature range might result in a poorer electrical conductivity due to temperature loss, which again may result in a slightly higher arcing voltage. The high thermal conductivity of air and the larger energy dissipation will result in heating of the surrounding gas. The gas surrounding the arc might act as an energy buffer, storing heat. Which is making the plasma channel more resistant to fast changes in temperature, setting higher demands for the cooling mechanism in the switchgear when using air instead of SF$_6$.

Even though the difference in thermal conductivity at high temperatures is a challenge when dealing with air instead of SF$_6$, it is the difference in the heat transport properties at lower temperatures that are the most challenging. At the moment right before CZ, it is an advantage that the thermal conductivity of the gas is high. This is because the current amplitude is low and falling, and due to the high thermal conductivity the energy in the plasma channel is transferred out to the surroundings fast. This results in a temperature drop in the arc. The speed of the temperature drop will depend on the temperature of the surroundings and the thermal conductivity of the gas. As the arc extinguishes and the current is zero, it is crucial that the ionized gas recombines to avoid re-ignition. The recombination speed is mainly dependent on the temperature of the gas. 

Since SF${_6}$ have a high thermal conductivity for temperatures that will occur around CZ the gas will cool down quickly and recombine fast. Air will however use longer time than SF${_6}$ to cool down and recombine, due to its low thermal conductivity at this stage in the interruption process. This means that in the moment right after CZ there will be a lot of ionized air particles between the contact plates, but in an SF${_6}$ based breaker more of the gas have recombined. This property will result in a higher chance of thermal re-ignition in an air-based breaker, and make stronger demands on the cooling mechanism.

\subsubsection*{Dielectric properties} 

Even though this research project is mainly focused on thermal re-strikes in a LBS a few things should be mentioned about the difference in dielectric properties between SF${_6}$ and air. Figure \ref{fig:breakDownVoltage} indicates the different breakdown voltages for SF$_6$ and air for a gas filled gap at 1 mm with a homogeneous electrical field. As the figure points out SF$_6$ has a much higher breakdown strength than air, approximately three times higher, but this depends on the gas pressure.

\begin{figure}[H]
\centering
\includegraphics[scale=1]{Bilder/Discussion/Breakdown_voltage.png}
\caption{Breakdown voltage at certain pressure levels of different gases in a homogeneous field with a gap space equal 1 mm  \cite{bib:TET4160HVIM}.} \label{fig:breakDownVoltage}
\end{figure}

The huge difference in breakdown voltage is mainly due to the high electron negative properties of SF$_6$. Electron negative gases are complex molecular structures mostly consisting of atoms from the halogen group in the periodic system, usually chlorine or fluorine \cite{bib:TET4160HVIM}. These atoms can easily capture free electrons because they lack one electron in the outer shell, and when capturing an electron they become negatively charged ions. Therefore it is possible to assume that in an electron negative gas the concentration of free electrons will be low, but due to ionisation of the gas the density of negative ions will be lager. However, the weight of the negatively charged gas molecule makes it less mobile than the much lighter free electron. This makes the impact from free electrons on breakdown voltage much higher than the impact from ionized molecules, resulting in a high breakdown voltage for electron negative gases.

Nitrogen is an electron positive gas because it tends to give away electrons from its outer shell. Due to the high concentration of nitrogen in air and the election positive effect, air's breakdown voltage becomes poorer than the one of SF$_6$.

\subsubsection*{Chemical properties}
Both air and SF$_6$ are chemically well suited for use as an interrupting gas. They are stable, non-toxic, non-flammable, and non-explosive. However, SF$_6$ forms highly toxic and corrosive compounds when subjected to electrical discharges \cite{bib:SF6PI}. If water vapour is present in an SF$_6$ gas the fluorine in SF$_6$ may react with the hydrogen in water, resulting in the formation of hydrofluoric acid \cite{bib:SF6PI}. Hydrofluoric acid is corrosive and will over time damage the switchgear. Fluorine gas and ionised fluorine also form during electrical discharges. Although highly toxic the gas is contained in the switchgear and therefore considered safe to use. But in the event of a blow-out, persons in close proximity to the switchgear or inside a building containing a large SF$_6$ insulated system might be exposed to fluorine gas and hydrofluoric acid. Air does represent the same toxicity hazard during a blow-out, but fragments and shockwaves from the explosion are in the same way as when using SF$_6$ a risk.

In cold climates, SF$_6$ might condensate and be partially liquified. This may result in a lower breakdown strength \cite{bib:SF6PI}. Air does not condensate under any normal operating temperatures. However, the problem with condensation of SF$_6$ is easily avoided with sufficient heating in the switchgear and therefore it is not considered a huge problem.

\newpage
\subsection{Environmental impacts of SF$_6$ from electrical power industries} \label{sec:EnvirImp}
%Locally the environmental impact from SF$_6$ based switchgear is low. If contaminants, like air and water vapour, is present in the switchgear highly toxic and corrosive compounds might form when the SF$_6$ is subjected to electrical discharges \cite{bib:SF6PI}. Usually this will only be harmful for the equipment, but can result in danger for personnel in close proximity of the switchgear or inside a building containing a large SF$_6$ insulated system during a blow-out. However SF$_6$ is considered to be a safe and stable compound for use in switchgear. The main environmental impact of SF$_6$ is based on its potential as an green house gas.

SF$_6$ is a highly efficient infrared absorber, and this combined with its chemical inertness makes it one of the strongest greenhouse gases \cite{bib:SF6PI}. In Norway, SF$_6$ makes up 0.4 \% of the total greenhouse gas emissions when measured in CO$_2$ equivalents \cite{bib:KlimaKur2020}. Due to the greenhouse gas potential of SF$_6$ this is a fairly small amount of gas, and the emissions is mostly from leakage in high voltage equipment. In Norway, the use of SF$_6$ is regulated through a voluntary agreement between the user group
and the environmental department \cite{bib:KlimaKur2020}. The user group consists of almost all major hydropower companies and major electricity distribution companies, but not all owners of SF$_6$ based equipment takes part in the agreement between the user group and the environmental department.


\begin{figure}[H]
\centering
\includegraphics[scale=0.4]{Bilder/Theory/consentrationSF6.png}
\caption{Average SF$_6$ concentration in the atmosphere \cite{bib:consSF6}.} \label{fig:conSF6}
\end{figure}

Because of the increase in commercial of use SF$_6$ since the 1970s the production of the gas has steadily increased. This have resulted in a rise of the SF$_6$ concentration in the atmosphere from barely measurable quantities in the 1980s \cite{bib:SF6PI} to relatively high quantities today. In figure, \ref{fig:conSF6} the concentration of SF$_6$ in the atmosphere from 1998 and upon till today is indicated. Awareness of SF$_6$ as a potent greenhouse gas has increased in recent years, and as figure \ref{fig:SF6EmissNor} illustrates the emissions have been reduced by almost a half in the period from 2000 to 2003. The voluntary agreement was signed in 2002 and resulted in a methodological change in how emissions was reported in 2003 \cite{bib:regSF6Miljo}. Reports from the industry points at a significant improvement in the handling of SF$_6$ \cite{bib:StatSF6}. It is these two major changes that are reasons as to the apparently huge drop in SF$_6$ emissions from the power industry in Norway between 2000 and 2003.

\begin{figure}[H]
\centering
\includegraphics[scale=0.6]{Bilder/Theory/emissionsSF6Norway.png}
\caption{Actual emission of SF$_6$ in Norway \cite{bib:StatSF6}.} \label{fig:SF6EmissNor}
\end{figure}

In 2005, the Norwegian SF$_6$ bank consisted of 202 tonnes of gas. Mostly installed in high voltage switchgear and circuit breakers within the user group members and in addition an estimated 2 tonnes were installed non-members equipment \cite{bib:StatSF6}. Medium voltage switchgear is mainly used by distribution companies and the installed capacity is estimated to be 60 tonnes in 2000 \cite{bib:StatSF6}, of which the user group controls half, and the rest is controlled by non-members.

\begin{table}[H]
\center
\caption{Leakage rates from product containing SF$_6$ \cite{bib:StatSF6}.}
\begin{tabular}{|c|c|c|}
\hline 
\textbf{Product emission source}
 & \textbf{Yearly rate of
leakage (per cent)}
 & \textbf{Product lifetime
(years)}
 \\ 
\hline 
Gas-insulated switchgear (GIS)
 & 1 & 30 \\ 
\hline 
Sealed medium voltage switchgear
 & 0.1 & 30 \\ 
\hline 
Electrical transformers for
measurements
 & 1 & 30 \\ 
\hline 
Sound-insulating windows
 & 1 & 30 \\ 
\hline 
Footwear (trainers)
 & 25 & 9 \\ 
\hline 
\end{tabular} 
\label{tab:leakageRatesProdSF6}
\end{table}

Table \ref{tab:leakageRatesProdSF6} points out that most of the emissions into the atmosphere are from high voltage switchgear called GIS. This is due to the leakage rate and the installed bank of 202 tonnes of SF$_6$. A fairly low leakage rate and amount of SF$_6$ installed in medium voltage switchgear make greenhouse gas emissions from this post quite low compared to the GIS post.

Since only half of the SF$_6$ bank in medium voltage switchgear is regulated through the environmental department with the voluntary agreement it is possible that the government will be more active in regulation of these kinds of equipment if a more "green" technology becomes available. Taxation or partially banning this equipment to gain some control over the use might be done. But as pointed out in table \ref{tab:leakageRatesProdSF6} most of the SF$_6$ emissions in Norway is from gas-insulated switchgear, mostly used in high voltage installations. These emissions are hard to completely remove, and even if air might be possible to use in a LBS compact design, it is probably not possible to use in most of the high voltage gas-insulated systems. Therefore the impact in SF$_6$ emissions, even if completely phasing out SF$_6$ based medium voltage switchgear, might not be very high. However, it is one of several things that might be done to reduce the SF$_6$ emissions in Norway.

In Norway, handling and leakage control of SF$_6$ is quite good, especially when compared to some other countries. Therefore it can be argued that the biggest reduction in SF$_6$ emissions will be in other countries where SF$_6$ is not regulated in the same manner. However, this reduction depends on air-bases equipment being made available and taken to use. To make this happen it is important that air-based LBSes is equally good or better than its SF$_6$ counterpart to interrupt currents. It must also be economicly feasible regarding lifetime, productions, and maintenance costs. Some kind of international agreement to reduce SF$_6$ emissions might also make the air-based breaker more economicly feasible on the global market. With this in mind, it is possible to assume that the global SF$_6$ emissions will be reduced if a competitive air-based LBS is developed.

\newpage

\section{Method} \label{sec:Method}
Most of the information in section \ref{sec:Method} is collected from the article \textit{ "Air Flow Investigation for a Medium Voltage for a Medium Voltage Load Break Switch"} by N. S. Aanensen, E. Jonsson, and M. Runde \cite{bib:AFIMVLBA}.
\newline

\subsection{The Switch and Contact Geometry} \label{sec:switchAndContactGeo}
This experiment is conducted using copper-tungsten arcing contacts, polytetraflourelthylene (PTFE) nozzles and air as interrupting medium. It is an open system, with the surrounding air at atmospheric pressure \textit{p$_0$} and a six litre tank with a pre-filled upstream overpressure \textit{p$_u$}, used during the interruption process to quench the arc. It is possible to adjust the upstream pressure, contact speed and position at current zero (CZ) independently, as well as the contact and nozzle geometry. The current and transient recovery voltage (TRV) can be change by changing the parameters of the laboratory test circuit.

\begin{figure} [h]
\centering
\includegraphics[scale=0.3]{Bilder/Method/contactSetUp.png}
\caption{The contact and nozzle. The diameter of the contact is \textit{d} and the inner diameter of the nozzle is \textit{D}  \cite{bib:AFIMVLBA}.} \label{fig:contactAndNozzle}
\end{figure}


A simple drawing of the contact is displayed in figure \ref{fig:contactAndNozzle}. The length of the cylindrical nozzle is 30 mm and the inner diameter is \textit{D}. Two different contact geometries that is going to be tested and are denoted a and b. The dimensions of the different geometries are illustrated in table \ref{tab:contGeoPara} and how the different dimensions are defined is illustrated in figure \ref{fig:AreacontactAndNozzle}.

\begin{table}[H]
\center
\caption{Contact geometry parameters}
 \begin{tabular}{|c|c|c|c|c|c|c|}
\hline 
Geometry & D [mm] & d [mm] & $\frac{D}{d}$ & $A_{contact}$ [mm$^2$] & $A_{ring}$ [mm$^2$] & $A_{nozzle}$ [mm$^2$] \\ 
\hline 
a & 10.4 & 4 & 2.60 & 12.6 & 72.4 & 84.9 \\ 
\hline 
b & 8 & 3 & 2.67 & 7.07 & 43.2 & 50.3 \\ 
\hline 
\end{tabular} 
\label{tab:contGeoPara}
\end{table}

\begin{figure} [h]
\centering
\includegraphics[scale=0.3]{Bilder/Method/areaSetUp.png}
\caption{Overview over how the different areas and diameters are defined \cite{bib:AFIMVLBA}.} \label{fig:AreacontactAndNozzle}
\end{figure}

The contact position \textit{x} is defined as the axial distance between the tulip and the pin contact. At starting position, \textit{x}= -60 mm, the pin contact is acting as a plug for the tank. This is making it possible to set on upstream pressure. The contact is hold in place by a electromagnet and is set to motion when the magnet releases the contact by a compressed spring. The spring accelerates the pin contact up to a speed of approximately 5.5 m/s at \textit{x}=0. At this position the spring is unloaded  and the pin moves with a constant speed until the contact is fully open at \textit{x}=110 mm.

\subsection{Test circuit}

\begin{figure} [H]
\centering
\includegraphics[scale=0.3]{Bilder/Method/circuit.png}
\caption{Circuit used for the interruption test \cite{bib:AFIMVLBA}.} \label{fig:testCurcuit}
\end{figure}

Figure \ref{fig:testCurcuit} is the laboratory test circuit used for the interruption test and supplies 50 Hz / 13.8 kV . It is possible to shape the TRV with tuning the parameters: L$_1$, L$_s$, R$_1$, R$_d$ and C. The systems short circuit parameters are R$_{sc}$ and L$_{sc}$. The TRV generated during interruption is set to simulate the standard for a 24 kV / 630 A class from the International Electrotechnical Commission (IEC) witch corresponds to:

\begin{itemize}
\item The initial part of the TRV has a rate of rise in recovery voltage (RRRV) of 71 - 73 V / $\mu$s.
	\begin{itemize}
		\item voltage difference is measured over the first 20 $\mu$s after CZ.
	\end{itemize}
\item The first voltage peak is between 7.2 and 7.4 kV with a rise time of approximately 96 $\mu$s
\end{itemize}

\begin{table}[H]
\center
\caption{Circuit Parameters and Resulting Current \cite{bib:AFIMVLBA} }
\begin{tabular}{|c|c|c|c|c|c|}
\hline 
L$_s$ [mH] & L$_1$ [mH] & R$_1$ [$\Omega$] & C [nF] & R$_{d}$ [$\Omega$] & I [A] \\ 
\hline 
14.5 & 138.4 & 35.5 & 74 & 248 & 400 \\ 
\hline 
6.9 & 86.2 & 22.1 & 102 & 198 & 630 \\ 
\hline 
\end{tabular} 
\label{tab:testParameters}
\end{table}

In table \ref{tab:testParameters} the value of the different test circuit parameters and the corresponding current can be observed. The test are done at currents with an RMS value of 400 A and 630 A. In all the test the TRV is kept constant up to and including the first voltage peak. In the case of a failed interruption thermal re-ignition occurs within a few microseconds after CZ.

A resistive transducer is measuring the contact position while a Hall effect current transducer is measuring the current through the test switch. The voltage between the contacts is measured with a parallel resistive / capacitive voltage divider. All measurements are transmitted through optical fibres to a 12 bit resolution transient recorder with a sampling frequency is 2.5 MHZ. The pressure in the tank is only measured before each test with an accuracy of 0.01 bar.

\subsection{Procedure} \label{sec:procedure}
Interruption tests with CZ occuring both inside and outside the nozzle are carried out. Previous work with a similar setup found that the interruption capability is better outside the nozzle. With two current ratings a total of four cases have been tested, if counting interruptions inside and outside the nozzle as the same case. The first and second CZ is included in this study to provide as much data as possible.

Inside nozzle is defined as contact position x = [5, 25] mm and outside nozzle as x = [35, 60] mm at first CZ. Interruption tests with first CZ either in the vicinity of the female contact (x $<$ 5 mm) or in the boundary region between inside and outside the nozzle (x = 25, 35 mm) are not included in the "interruption success rate graphs" but are presented in tables displaying raw data. The first CZ occurred within x $<$ 60 mm and the second CZ occurred for x $>$ 60 mm, as the contact speed during all tests was 5.5 mm/ms $\pm$ 0.5 mm/ms.

The test procedure for each of the four cases as follows: 
\begin{itemize}
\item[1.] A pressure level that seems to be in the area of interest is found by performing some initial test interruptions at different pressure levels. This level is kept constant for ten interruption tests, five inside and five outside the nozzle.
\item[2.] If a pressure level results in less than ten out of ten successful interruptions, ten new tests with a higher upstream pressure (next level) is conducted. This is repeated until at least one pressure level with ten successful interruptions is found.
\item[3.] Then, the pressure is stepped down until seven or more out of ten interruption attempts fail or the lowest possible pressure level is tested.
\end{itemize}

The pressure level step is 0.1 bar for all tests and at least three pressure levels are tested for each case.\newline

The pin is cleaned, polished, and greased between each test to ensure a smooth surface. The contacts and nozzle are replaced regularly to avoid contact wear and nozzle deformation. This is to ensure that the geometry is constant through the whole experiment.

\section{Results}
\subsection{Interruption experiment}
Originally the intention with the experiments was to test if an equal air flow velocity when the pin was inside or outside the nozzle in the moment of interruption resulted in a equal interruption success rate. However this hypothesis was dismissed since the results from the initial testing indicated that the switch was unable to interrupt at all when the pin was inside the nozzle. Therefore the test procedure was rearranged and the focus was moved to the area when the pin was outside the nozzle at the moment of interruption. At least five interruption test was done at the relevant pressure levels outside the nozzle as described in section \ref{sec:procedure}. The goal with the experiment is to compare the different geometries to each other with regards to interruption success rate compared to pressure level and current amplitude.

For the interruption tests there are four outcomes, at the first CZ the interruption can succeed or a thermal re-ignition can occur. Given a thermal re-ignition in the first CZ the current can be interrupted at the second CZ or another thermal re-ignition can occur. The end result have two outcomes: success or failure. The outcomes is explained by figure \ref{fig:pilSuccessOfFail}. Besides the outcomes listed, there is a small chance of a dielectric re-ignition. However these are rare and are therefore not included. Experience has also shown that in practical interruption cases thermal re-ignitions are the most demanding tasks for load break switches in the 24 kV 630 A class \cite{bib:AFIMVLBA}. During testing of geometry \textit{b} at 630 A two re-ignitions occurred that was in the area between thermal and dielectric re-ignition, these was counted as thermal failures.

\begin{figure}[H]
\centering
\includegraphics[scale=0.7]{Bilder/Results/interruptionFlowChart.png}
\caption{Flowchart illustrating the different interruption scenarios.} \label{fig:pilSuccessOfFail}
\end{figure}

In figure \ref{fig:CurrentAndVoltageWaveform} the current and voltage during a unsuccessful and successful interruption are indicated for a 630 A 24 kV test. The upper plot illustrates a thermal re-ignition, as the plot shows; the current is almost unaffected of the interruption attempt and continues as normal after the CZ. The lower plot is illustrating a successful interruption, where the current have stopped flowing after CZ and a recovery voltage have raised between the contacts.

\begin{figure}[H]
\centering
\includegraphics[scale=0.3]{Bilder/Results/differentInterruptions.png}
\caption{Current and voltage waveforms near CZ for two different interruption outcomes \cite{bib:AFIMVLBA}.} \label{fig:CurrentAndVoltageWaveform}
\end{figure}

In figure \ref{fig:successRate400A} the results form the 400 A test for geometry \textit{a} and \textit{b} is presented. The graph are produced with the results from appendix \ref{app:testResults400A}. Figure \ref{fig:successRate630A} points out the results from the two geometries at 630 A, and it is produced from the data in appendix \ref{app:testResults630A}. The data from the boundary regions, which is when the pin are between 25 mm to 35 mm and less than 5 mm away from the female contact, is removed form the data selection. Minimum five tests where done at each pressure level, and the total chance of interruption are represented as a point in the graph. The exact number of tests at each point is pointed out in the raw data in appendix \ref{app:testResults400A}. It should be noted that in the higher pressure levels only a one or two tests at each pressure was done to see if the geometry was able to interrupt currents when the pin was inside the nozzle. Therefore higher variation is expected in the high pressure levels.  

\begin{figure}[H]
\centering
\includegraphics[scale=0.8]{Bilder/Results/successRate400A.png}
\caption{Interruption success rate for geomerty \textit{a} and \textit{b} at 400 A.} \label{fig:successRate400A}
\end{figure}

For the 400 A experiment, geometry \textit{b} performed a little better than geometry \textit{a}. Geometry \textit{b} manage to interrupt 100 \% of the tests at a pressure of 0.7 bar, while geometry \textit{a} did this on 0.8 bar. However geometry \textit{a} manage to interrupt 90 \% of the tests at 0.7 bar, resulting in a almost equal performance for both geometries at 400 A.

\begin{figure}[H]
\centering
\includegraphics[scale=0.5]{Bilder/Results/successRate630A.png}
\caption{Interruption success rate for geomerty \textit{a} and \textit{b} at 630 A.} \label{fig:successRate630A}
\end{figure}

For the 630 A experiment, both geometries successfully interrupted 100 \% of the tests at 1.2 bar. However geometry \textit{a} was more stable and more predictable for each pressure level than geometry \textit{b}. Geometry \textit{b} had more variation in the number of successful interruptions for each pressure level. The variation might have been due to the small dimension of the contact compared to the high amplitude of the current.

Since only a limited amount of tests have been carried out at each pressure level it is important not to confuse the results presented in this sections as statistical proved limits. The results can be regarded as a tool to use in future experiments and may indicate a possible new test geometry. The results can indicate at what pressure level the geometry was able to interrupt or not, but high uncertainties are connected to the percent of successful interruptions and it should not be used as empirical evidence.

The raw data presented in appendix \ref{app:rawData} are presented without alterations. The black line indicates the length of the nozzle and the boundary regions are illustrated with the grey areas. If several tests occurred at the same position in the nozzle, the marker indicating the result have been moved a bit to the side so that the number of tests at each pressure level is countable. For the 400 A experiment the position measurer was damaged and resulted in an inaccuracy of $\pm$ 2.5 mm for each result. In the 630 A experiment the the accuracy for the position is set to $\pm$ 1 mm.

\subsection{Arcing voltage}
During the experiments the arcing voltage have been monitored and is presented below in figure \ref{fig:arcingVoltage400A} and \ref{fig:arcingVoltage630A}. Each graph illustrates the arcing voltage at a failed interruption when the pin was inside the nozzle at a given pressure. Voltage is indicated along the y-axis and the x-axis indicates time. When CZ occurs the voltages changes polarities and what was before the arcing voltage becomes the TRV. Each voltage plot is shifted along the x-axis a bit so that it is possible to spot the difference between them.

Figure \ref{fig:arcingVoltage400A} displays the arcing voltage for geometry \textit{a} at 400 A. While figure \ref{fig:arcingVoltage630A} illustrates the arcing voltage for geometry \textit{a} at 630 A. There is only one interruption test for each pressure level.

\begin{figure}[H]
\centering
\includegraphics[scale=0.5]{Bilder/Results/arcingVoltage400Ad4.png}
\caption{Arcing voltage for geomerty \textit{a} at 400 A for different pressure levels.} \label{fig:arcingVoltage400A}
\end{figure}

\begin{figure}[H]
\centering
\includegraphics[scale=0.5]{Bilder/Results/arcingVoltage630Ad4.png}
\caption{Arcing voltage for geomerty \textit{a} at 630 A for different pressure levels.} \label{fig:arcingVoltage630A}
\end{figure}



\subsection{Durability of the arcing contacts} \label{sec:durability}

Since the durability of the arcing contacts was fund to be dependent on the current amplitude some picture of the arcing contacts before and after use is displayed below. A new unused male contact is displayed in figure \ref{fig:unused_d3}.

\begin{figure}[H]
\centering
\includegraphics[scale=0.3]{Bilder/Discussion/d3_unused.png}
\caption{Unused male arcing contact of copper-tungsten.} \label{fig:unused_d3}
\end{figure}

For the low current test, with an RMS value of 400 A, both contact diameters handled the wear quite well. The pin that had a diameter of 4 mm was exposed to 82 interruptions. The same pin was also used at the high current test, with an RMS value of 630 A, and was exposed to 37 interruptions. The material in the pin handled this quite well, all though some sign of damage was observed this did not seem to have had an impact on the interruption capabilities or the mechanical strength of the pin. The pin after the tests is displayed in figure \ref{fig:d4_burn_side} and \ref{fig:d4_burn_top}. The pin was exposed to both successful and unsuccessful interruptions, therefore a mix of both short and long arcing times.

\begin{figure}[H]
\centering
\begin{minipage}{.5\textwidth}
  \centering
  \includegraphics[scale=0.2]{Bilder/Discussion/d4_630and400_burn.png}
  \caption{Male contact with a diameter of 4 mm, \newline side view.}
  \label{fig:d4_burn_side}
\end{minipage}%
\begin{minipage}{.5\textwidth}
  \centering
  \includegraphics[scale=0.53]{Bilder/Discussion/d4_630and400_top_burn.png}
  \caption{Male contact with a diameter of 4 mm, \newline top view.}
  \label{fig:d4_burn_top}
\end{minipage}
\end{figure}

The smallest geometry, with a diameter of 3 mm, handled the 400 A test in the same manner as the pin with a diameter of 4 mm. It had signs of wear, but did not needed to be changed during the test and its wear is not assumed to have had an impact on the test result. The smallest geometry is displayed in figure \ref{fig:d3_burn_side} and \ref{fig:d3_burn_top}. This pin was used for five unsuccessful interruptions before it needed to be replaced.


\begin{figure}[H]
\centering
\begin{minipage}{.5\textwidth}
  \centering
  \includegraphics[scale=0.2]{Bilder/Discussion/d3_630_burn.png}
  \caption{Male contact with a diameter of 3 mm, \newline side view.}
  \label{fig:d3_burn_side}
\end{minipage}%
\begin{minipage}{.5\textwidth}
  \centering
  \includegraphics[scale=0.53]{Bilder/Discussion/d3_630_top_burn.png}
  \caption{Male contact with a diameter of 3 mm, \newline top view.}
  \label{fig:d3_burn_top}
\end{minipage}
\end{figure}

Figure \ref{fig:d3_burn_top} illustrates that droplets of copper that have vaporised from the tip and condensed longer down on the pin. The tip of the pin is severely disfigured and only parts of the tungsten skeleton remains while most of the copper has vaporised from the pin. The results from this test have a lot more variation than the other tests and it might be partially due to the fast wear down of the pin.  


\begin{figure}[H]
\centering
\includegraphics[scale=0.4]{Bilder/Discussion/femaleContacts4mmand3mm.png}
\caption{Used female arcing contact of copper-tungsten. \newline The one to the left has a diameter of 4 mm and the one to the right 3 mm.} \label{fig:used_d4_d3_female}
\end{figure}

As illustrated in figure \ref{fig:used_d4_d3_female} the female contact for both geometries have few signs of wear. Non of the female contacts were replaced during the experiment, but was polished half way trough each test series to remove soot and other pollution that may have formed on them. The surface area which and arc can burn on is larger for the female contact than the male contact and might be the reason for the much higher durability in the female contact than the male contact.

\newpage

\section{Discussion}

 
\subsection{arcing voltage and pressure}

\subsection{air pressure, speed and volume flow}

\subsection{Durability of the arcing contacts} 

Due to harsh condition during the interruption process it is important that the different parts of the switch is able to tolerate the stress of an arc for several breaking operations and for at least two current zero crossings before the arc is extinguished. As briefly described in section \ref{sec:genDes} and \ref{sec:switchAndContactGeo} the arcing contacts in a LBS consists of different metals or composites designed to to withstand the stress from the arc. A common material for the arcing contact is a pseudo-alloy of copper and tungsten called copper-tungsten, this material is also used in the contact-set of the test switch. The main advantage of this pseudo-alloy is that it is able to conduct current quite well because of the good conductivity of the copper, however copper alone will evaporate if exposed to an arc. Tungsten have a high boiling point, and will not vaporise with the same speed as the copper, this makes the contacts more resistant to the stress of the arc. Of the two geometries tested there was a significant difference in the material durability in the male contact.

The results presented in \ref{sec:durability} might indicate that a small amount of copper-tungsten can be used in the arcing contacts in a commercial LBS, which is good to minimize costs. However it is highly dependent on the current amplitude since the cross-section and temperature of the arc is highly dependent on the current magnitude. It is the cross-section and temperature of the arc that is assumed to have the biggest impact of the wear in the contact pins. 

Geometry \textit{b} did not tolerate the high current test, and quickly degenerated as displayed in figure \ref{fig:d3_burn_side} and \ref{fig:d3_burn_top}. If figure \ref{fig:tempDist2} is consulted it might indicate that the temperature in the arc have a significant rise from the test with an current amplitude of 400 A to the tests with an amplitude of 630 A. However this increase in the arcs temperature will be experienced by both contact geometries and does not alone explain the significant difference in wear. But geometry \textit{b} will have a smaller surface area where the arc can move along and this difference combined with the increased arcing temperature might have resulted in a faster vaporisation of copper and degeneration of the male contact for geometry \textit{b}.

Test where the current was interrupted every time was also done with geometry \textit{b}. The degeneration of the male contact was then slower and less severe. Therefore it might be argued that if a interruption can be guaranteed at the first CZ geometry \textit{b} might also be used to interrupt larger currents. But it should be noted that even if it is possible to successfully interrupt the current on a regular basis without wearing down the pin, a small pin geometry, like geometry \textit{b}, might not be dielectric suitable, especially if air is used instead of SF$_6$ as interrupting medium. A small round shape might be to sharp and can give an enhancement of the electrical field around the tip, resulting in partial discharges and a lower breakdown voltage for the design. This might give a problem with spark-overs when the switch is open, due to transients in the power grid, or result in a too long arcing time when closing the switch. Both incidents can be harmful to a fragile arcing contact and may shorten the lifetime of the LBS.
\newpage
\subsection{Suggestion for further work}

\section{Conclusion}

\newpage
\begin{thebibliography}{10}


\bibitem{bib:HVEbreak} \textit{M. Runde}, \textit{Current Interruption in Power Grids}. Trondheim: Norwegian University of Science and Technology, 2013

\bibitem{bib:SF6PI} \textit{L.G. Christophorou, J. K. Olthoff, and R.J. Van Brunt}, "\textit{Sulfur Hexafluoride and the Electric Power Industry}", \textit{IEEE Electrical Insulation Magazine, vol. 13, No. 5, pp. 20-24}, Oct. 1997.

\bibitem{bib:comSub} \textit{amesimpex.com}, \url{http://www.amesimpex.com/images/unitised_sub_002.jpg}, \textit{26.9.2013}

\bibitem{bib:CIAMVLBS} \textit{E. Jonsson, N. S. Aanensen and M. Runde}, "\textit{Current Interruption in Air for a Medium Voltage Load Break Switch}", \textit{IEEE Trans. Power Delivery}, to be published.

\bibitem{bib:KlimaKur2020} "\textit{KLIMAKUR2020}", Oslo: Klima- og forurensningsdirektoratet, 2010

\bibitem{bib:CBAC} \textit{W. Rieder}, "\textit{Circuit breakers, Physical and engineering problems, III-Arc-medium considerations}", \textit{IEEE spectrum, pp. 80-84}, Sept. 1970.

\bibitem{bib:IPSF6AQM} \textit{W. Hertz, H. Motschmann and H. Wittel}, "\textit{Investigations o the Properties of SF$_6$ as an Arc Quenching Medium}", \textit{Proceedings of The IEEE, vol. 59, NO. 4, pp. 485-492}, April 1971.

\bibitem{bib:TDCIGBB} \textit{W. Hermann}, "\textit{THEORETICAL DESCRIPTION OF THE CURRENT INTERRUPTION IN HV GAS BLAST BREAKERS}", \textit{IEEE Transactions on Power Apparatus and System, vol. PAS-96, NO. 5, pp. 1546-1555}, Sept./ Oct. 1977.

\bibitem{bib:consSF6} \textit{esrl.noaa.gov}, \url{http://www.esrl.noaa.gov/gmd/webdata/iadv/ccgg/graphs/pdfs/ccgg.MLO.sf6.1.none.discrete.all.pdf}, \textit{17.10.2013}

\bibitem{bib:StatSF6} \textit{K. L. Hansen}, "\textit{Emissions from consumption of HFCs, PFCs and SF$_6$ in Norway}", \textit{Statistics Norway/Department of Economic Statistics/Environmental Statistics
}, 2007.

\bibitem{bib:regSF6Miljo} \textit{regjeringen.no}, \url{http://www.regjeringen.no/nb/dep/md/dok/regpubl/stmeld/2011-2012/meld-st-21-2011-2012/5/5.html?id=682932}, \textit{21.10.2013}



\bibitem{bib:TET4160HVIM} \textit{E. Ildstad}, "\textit{High Voltage Insulation Materials}", \textit{Trondheim: Norwegian University of Science and Technology, 2012
}, August 2012.

\bibitem{bib:THFD} \textit{R. W. Johnson}, "\textit{The Handbook of FLUID DYNAMICS}", \textit{Heidelberg: Springer-Verlag GmbH \& Co. KG
}, 1998.

\bibitem{bib:AFIMVLBA} \textit{N. S. Aanensen, E. Jonsson, and M. Runde}, "\textit{Air Flow Investigation for a Medium Voltage Load Break Switch}", \textit{IEEE Trans. Power Delivery}, to be published.

\end{thebibliography}

\appendix
\section{Test Results} \label{app:rawData}
\subsection{400 A geometry \textit{a} and \textit{b}} \label{app:testResults400A} 
\begin{figure}[H]
\centering
\includegraphics[scale=0.55]{Bilder/Results/rawData400AgeoA.png}
\caption{Raw data for geometry \textit{a} at 400 A, The grey areas are the boundary region where results was discarded for the success rate graphs. The black line indicates where the length of the nozzle.} \label{fig:rawData400AgeoA}
\end{figure}
\newpage

\begin{figure}[H]
\centering
\includegraphics[scale=0.55]{Bilder/Results/rawData400AgeoB.png}
\caption{Raw data for geometry \textit{b} at 400 A, The grey areas are the boundary region where results was discarded for the success rate graphs. The black line indicates where the length of the nozzle.} \label{fig:rawData400AgeoA}
\end{figure}
\newpage

\subsection{630 A geometry \textit{a} and \textit{b}} \label{app:testResults630A}
\begin{figure}[H]
\centering
\includegraphics[scale=0.55]{Bilder/Results/rawData630AgeoA.png}
\caption{Raw data for geometry \textit{a} at 630 A, The grey areas are the boundary region where results was discarded for the success rate graphs. The black line indicates where the length of the nozzle.} \label{fig:rawData400AgeoA}
\end{figure}
\newpage

\begin{figure}[H]
\centering
\includegraphics[scale=0.55]{Bilder/Results/rawData630AgeoB.png}
\caption{Raw data for geometry \textit{b} at 630 A, The grey areas are the boundary region where results was discarded for the success rate graphs. The black line indicates where the length of the nozzle.} \label{fig:rawData400AgeoA}
\end{figure}


\end{document}
